\documentclass{assignment}

\coursetitle{Design and Analysis of Algorithms 1}
\courselabel{CPSC 413}
\exercisesheet{Home Work \#3}{Individual submission version}
\student{Tom Crowfoot - 10037477}
\semester{Winter 2014}
\usepackage{amsmath}
\usepackage[boxed]{algorithm2e}

\begin{document}
\begin{problemlist}
\pbitem
\begin{problem}
\end{problem}
\begin{answer}
\\
For (a):\\
i. $1\rightarrow 2\rightarrow 3\rightarrow 4\rightarrow 6\rightarrow 5\rightarrow 7\rightarrow 8$\\
ii. $1\rightarrow 2\rightarrow 3\rightarrow 4\rightarrow 6\rightarrow 5\rightarrow 7\rightarrow 8$\\
iii. Yes, $G$ is connected. To make it unconnected, remove the edge between nodes 4 and 6.\\
iv. Yes, $G$ contains a cycle. One example cycle is $1\rightarrow 2\rightarrow 3\rightarrow 1$.\\
v. No, $G$ is not a tree because it contains a cycle.\\
\\For (b):\\
i. A DFS tree of a completed graph would be one path throughout the entire tree.\\
ii. A BFS tree of a complete graph would have an edge between the root and every other node.\\
\\For (c):\\
You could create a one dimensional array, and fill it in with the values for each adjacent node. Then you could binary search this array in $logn$ time to determine adjacency.\\
\\For (d):\\
True.\\
Assume that there is a graph $G$ with $n$ nodes and every node has $deg(n/2)$, and that $G$ is not connected. Then for $v\in G$, $v$ must be connected to atleast $n/2$ other nodes. By assumption that $G$ is not connected, there must be a node $x$ that is not connected to $v$. As there are at most $n/2 -1$ nodes not connected to $v$, and there must be $n/2$ nodes connected to $x$, there must be an edge connecting $x$ to something that is connected to $v$, which is a contradiction.\\
\end{answer}
\clearpage
\pbitem
\begin{problem}
\end{problem}
\begin{answer}
\\
Input: An undirected graph $G$.\\
Output: Either that the graph does not contain a cycle, or a single cycle that was found.\\
Algorithm:\\
Pick a node in the graph.\\
Start doing breadth first search through graph.\\
If you come across a node that you have already visited, you have a cycle.\\
If you have visited every node in the graph, there is no cycle.\\
The algorithm is correct, and runs in $O(m+n)$ because it is exactly the same as the breadth first search.\\
\end{answer}
\clearpage
\pbitem
\begin{problem}
\end{problem}
\begin{answer}
\\
For (a):\\
It has 6.\\
a,b,c,d,e,f\\
a,b,d,c,e,f\\
a,b,d,e,c,f\\
a,d,e,b,c,f\\
a,d,b,e,c,f\\
a,d,b,c,e,f\\
\\For (b):\\
Precondition: the graph is a DAG, of size n\\
Post condition: a valid topological ordering for the DAG\\
\begin{algorithm}
  FindTO(G, n):\\{
    compute degreeincoming for each node\;
    push all nodes with degreeincoming == 0 onto stack\;
    add all nodes to arrayTO inorder\;
    \While{stack is not empty}{
      pop the stack = v\;
      add v to arrayTO\;
      \For{each edge between v and x}{
        degreeincoming(x)-1\;
        \If{degreeincoming(x)==0}{
          push(x)\;
        }
      }
    }
    return arrayTO\;
  }
\end{algorithm}
\\
step one will cost n, as it must work for each node in the graph.\\
setp two is constant\\
step three is same as two\\
the while loop will iterate n times, once for each node in the DAG\\
poping stack is constant, as well as adding element to array\\
the for loop will iterate m times, once for each edge\\
subtraction is constant, as is conditional and push\\
return is constant\\
so $n + c + \sum_{i=1}^n {c + m(c)} + c$\\
$n + \sum_{i=1}^n {c + m(c)} + c$\\
$n(c + mc) + n + c$\\
$nc + mc + n + c$\\
$n(c+1)+mc+c$\\
$n+m$\\
\\For (c):\\
Precondition: the graph is a directed graph, of size n\\
Post condition: a valid topological ordering for the DAG\\
\begin{algorithm}
  FindTO(G, n):\\{
    compute degreeincoming for each node\;
    push all nodes with degreeincoming == 0 onto stack\;
    add all nodes to arrayTO inorder\;
    \While{stack is not empty}{
      pop the stack = v\;
      add v to arrayTO\;
      \For{each edge between v and x}{
        degreeincoming(x)-1\;
        \If{degreeincoming(x)==0}{
          push(x)\;
        }
      }
    }
    \eIf{number of elements in arrayTO == n}{
      return arrayTO\;
    }
    \Else{      
      call algorithm from question two\;
    }
  }
\end{algorithm}
\\
proof of runtime:\\
it is a mash of two algorithms\\
first one is $O(n+m)$ and second one is also $O(m+n)$\\
so this one is $n+m +n+m= 2n+2m\Rightarrow O(m+n)$\\
the algorithm is correct because the two algorithms it uses are correct.
the modification will allow it to be non asyclic, in which case the while loop will terminate before all n nodes have been examined. thus the number of elements in arrayTO will be less than n, and it will call the cycle finding algorithm.\\
\end{answer}

\end{problemlist}
\end{document}
