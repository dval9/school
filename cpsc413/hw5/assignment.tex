\documentclass{assignment}

\coursetitle{Design and Analysis of Algorithms 1}
\courselabel{CPSC 413}
\exercisesheet{Home Work \#5}{}
\student{Julie Katayama - 10092143\\
Etienne Pitout - 10075802\\
Tom Crowfoot - 10037477\\
Thomas King - 10074105}
\semester{Winter 2014}
\usepackage{amsmath}
\usepackage{amssymb}
\usepackage{amsthm}
\usepackage[linesnumbered,noline]{algorithm2e}
\usepackage{mathtools}
\DeclarePairedDelimiter\floor{\lfloor}{\rfloor}
\DeclarePairedDelimiter\ceil{\lceil}{\rceil}

\begin{document}
\begin{center}
\renewcommand{\arraystretch}{2}
\begin{tabular}{|c|c|c|} \hline
Problem & Marks \\ \hline \hline
1 & \\ \hline
2 & \\ \hline
3 & \\ \hline
4 & \\ \hline \hline
Total & \\ \hline
\end{tabular}
\end{center}

\bigskip

\begin{problemlist}

\clearpage
\pbitem
\begin{problem}
Consider the following recurrence relation:
\begin{align*}
T(n)=T(\floor{n/2})+T(\floor{n/4})+T(\floor{n/8})+n.
\end{align*}
\begin{enumerate}
\item Use the iteration method to come up with a good guess for a tight asymptotic bound for the recurrence.
\item Prove that your guess is correct.
\end{enumerate}
\end{problem}
\begin{answer}
\\
\begin{enumerate}
\item
\begin{align*}
T(n)=&T(\floor{n/2}) + T(\floor{n/4}) + T(\floor{n/8}) + n\\
\le&T(n/2) + T(n/4) + T(n/8) + n\\
\le&T(7n/8) + n\\
=&T(7^2n/8^2) + 7n/8 + n\\
=&T(7^3n/8^3) + 7^2n/8^2 + 7n/8 + n\\
\Rightarrow&\sum_{i=0}^\infty (7/8)^in\\
=&(\frac{1}{1-\frac{7}{8}})n\\
\end{align*}
A guess for the tight asymptotic bound is $\Theta (n)$.\\
\item
Claim: $T(n)=T(\floor{n/2})+T(\floor{n/4})+T(\floor{n/8})+n \le c*n$.\\
Base case: If $1 \le n \le 1000$, then $T(n) = t \le c*n$.\\
We can then pick $c=t$.\\
Inductive Hypothesis: $T(k)=T(\floor{k/2})+T(\floor{k/4})+T(\floor{k/8})+k \le c*k$, for some $k < n$.\\
\begin{align*}
T(n)=&T(\floor{n/2}) + T(\floor{n/4}) + T(\floor{n/8}) + n\\
\le&a*\floor{n/2} + a*\floor{n/4} + a*\floor{n/8} + n\\
\le&a*(n/2) + a*(n/4) + a*(n/8) + n\\
=&a*(7n/8) + n\\
=&(7a/8 + 1) * n\\
\le&c*n\\
\end{align*}
If we choose $c=(7a/8 + 1)$, then we find $T(n)\in O(n)$.\\

Claim: $T(n)=T(\floor{n/2})+T(\floor{n/4})+T(\floor{n/8})+n \ge c*n$.\\
Base case: If $1 \le n \le 1000$, then $T(n) = t \ge c*n$.\\
We can then pick $c=t/1000$.\\
Inductive Hypothesis: $T(k)=T(\floor{k/2})+T(\floor{k/4})+T(\floor{k/8})+k \ge c*k$, for some $k < n$.\\
\begin{align*}
T(n)=&T(\floor{n/2}) + T(\floor{n/4}) + T(\floor{n/8}) + n\\
\ge&a*\floor{n/2} + a*\floor{n/4} + a*\floor{n/8} + n\\
\ge&a*(n/2) + a*(n/4) + a*(n/8) + n\\
=&a*(7n/8) + n\\
=&(7a/8 + 1) * n\\
\ge&c*n\\
\end{align*}
If we choose $c=(7a/8 + 1)$, then we find $T(n)\in \Omega(n)$.\\
As $T(n)\in \Omega(n)$ and $T(n)\in O(n)$, we can conclude $T(n)\in \Theta (n)$.\\
\end{enumerate}
\end{answer}

\clearpage
\pbitem
\begin{problem}
Use the Master Theorem to obtain tight asymptotic bounds on the following recurrences:
\begin{enumerate}
\item $T(n)=2T(\frac{n}{2})+n^3$.
\item $T(n)=2T(\frac{n}{2})+n$.
\item $T(n)=2T(\frac{n}{2})+\sqrt{n}$.
\item $T(n)=4T(\frac{n}{8})+\sqrt{n}lg^2~n$.
\item $T(n)=16T(\frac{n}{7})+n^2$.
\end{enumerate}
\end{problem}
\begin{answer}
\\
\begin{enumerate}
\item
Using case 3 of the Master theorem, $T(n)=2T(n/2)+n^3$, $a=2$, $b=2$, $log_22=1$, $f(n)=n^3=\Omega (n^{1+\epsilon})$, $0< \epsilon \le 2$, $2(n^3/8)\le cn^3$, $c=1/4$.\\
Thus, $T(n) = \Theta(f(n)) = \Theta(n^3)$.\\ 
\item
Using case 2 of the Master theorem, $T(n)=2T(n/2)+n$, $a=2$, $b=2$, $log_22=1$, $f(n)=n=\Theta (n^1)$.\\
Thus, $T(n) = \Theta(nlogn)$.\\
\item
Using case 1 of the Master theorem, $T(n)=2T(n/2)+\sqrt(n)$, $a=2$, $b=2$, $log_22=1$, $f(n)=\sqrt{n}=O(n^{1-\epsilon})$, $0<\epsilon \le 1/2$.\\
Thus, $T(n)=\Theta(n)$.\\
\item
Using case 1 of the Master theorem, $T(n)=4T(n/8)+\sqrt{n}log^2n$, $a=4$, $b=8$, $log_84=2/3$, $f(n)=\sqrt{n}log^2n$\\
\begin{align*}
&\lim_{n\rightarrow \infty}\frac{\sqrt{n}log^2n}{n^{2/3}}=\lim_{n\rightarrow \infty}\frac{log^2n}{n^{1/6}}=\lim_{n\rightarrow \infty}\frac{\frac{2logn}{n}}{\frac{1}{6}n^{-5/6}}\\
=&12\lim_{n\rightarrow \infty}\frac{logn}{n^{1/6}}=12\lim_{n\rightarrow \infty}\frac{n^{-1}}{\frac{1}{6}n^{-5/6}}=72\lim_{n\rightarrow \infty}\frac{1}{n^{1/6}}\\
=&0\\
\end{align*}
Using L'Hopitals rule, we see that $f(n)=o(g(n))$, so $f(n)=\sqrt{n}log^2n=O(n^{2/3-\epsilon})$, $0<\epsilon$.\\
Thus, $T(n)=\Theta (n^{2/3})$.\\
\item
Using case 3 of the Master theorem, $T(n)=16T(n/7)+n^2$, $a=16$, $b=7$, $f(n)=n^2=\Omega (n^{log_7 16 + \epsilon})$, $0<\epsilon \le 2-log_716$, $16(n^2/49)\le cn^2$, $k=16/49$.\\
Thus, $T(n)=\Theta (n^2)$.\\
\end{enumerate}
\end{answer}

\clearpage
\pbitem
\begin{problem}
Consider the problem of multiplying two $n\times n$ matrices where $n$ is not a power of 2.
\begin{enumerate}
\item Describe how Strassen's Algorithm can be modified to handle this case.
\item Prove that your modified algorithm runs in time $\Theta(n^{lg~7})$.
\end{enumerate}
\end{problem}
\begin{answer}
\\
\begin{enumerate}
\item
The modification will be to increase the size of the matrices that are being multiplied from $n\times n$, where $n \neq 2^i,~i\in \mathbb{Z}$ to a $m\times m$ matrix, such that $2^{i-1}< n < 2^i = m$, and all the new elements $a_{ij}=0$, $n < i,j < m$. After multiplying these two modified matrices, the result is a $m\times m$ matrix. The top left $n\times n$ elements can then be extracted from the resulting $m\times m$ matrix to get the desired result.\\
\item
By assumption $n \neq 2^i,~i\in \mathbb{Z}$, so let $2^{i-1}< n < 2^i = m$. We also find the relation $2*2^{i-1}< 2*n \Rightarrow 2^i< 2n$, which gives us the total relation $2^{i-1}< n < 2^i = m < 2n$. So the runtime of our $m\times m$ algorithm will be $\Theta (m^{lg~7}) < \Theta ((2n)^{lg~7}) = \Theta (2^{lg~7}n^{lg~7}) = \Theta (n^{lg~7})$. Thus, the runtime of the algorithm after modification is still the same.\\
\end{enumerate}
\end{answer}

\clearpage
\pbitem
\begin{problem}
Let $X$ and $Y$ be two arrays, each containing $n$ distinct integers in ascending order. The problem is to compute the median of all $2n$ elements in time $\Theta(lg~n)$.
\begin{enumerate}
\item Give a formal definition (pre- and post-conditions) of the problem described.
\item Give a divide-and-conquer algorithm for solving the problem.
\item Prove that your algorithm is correct.
\item Express the run-time of your algorithm as a recurrence relation. Explain why your recurrence relation is correct.
\item Prove a tight asymptotic bound on the run-time of your algorithm (proving upper and lower bounds separately if necessary). You may use the master theorem, or a version of the ``guess and prove" method.
\end{enumerate}
Note: Algorithms that do not use divide-and-conquer will recieve no credit. Divide-and-conquer algorithms with runtime in $\omega(lg~n)$ will receive partial credit.
\end{problem}
\begin{answer}
\\
\begin{enumerate}
\item
Pre-conditions: Two arrays $X$ and $Y$, which are sorted in ascending order. All elements in $X$ are distinct and all elements in $Y$ are distinct. $X$ contains $n$ elements and $Y$ contains $n$ elements.\\
Post-conditions: An integer $b$, such that $b \in X \cup Y$, and $b$ is the median of $X \cup Y$.\\
\item
Algorithm to find the median of two sorted arrays, $X$ and $Y$, of length $n$.\\
\IncMargin{3em}
\begin{algorithm}
  int median(X, Y, n)\\\{\\
  \Indp
  m1 = X[$\ceil{n/2}$]\;
  m2 = Y[$\ceil{n/2}$]\;
  \uIf{$n == 1$}{
    \Return{min(X[0], Y[0])}\;
  }
  \uElseIf{$m1 == m2$}{
    \Return{m1}\;
  }
  \uElseIf{$m1 > m2$}{
    \Return{median(X[0..$\ceil{n/2}$], Y[$\ceil{n/2}$..n], $\ceil{n/2}$)}\;
  }
  \uElseIf{$m1 < m2$}{
    \Return{median(X[$\ceil{n/2}$..n], Y[0..$\ceil{n/2}$], $\ceil{n/2}$)}\;
  }
  \Indm
  \}
\end{algorithm}
\DecMargin{3em}
\item
Base case: $n=1$.\\
If $n=1$, then both arrays only have one element. The median will then be the minimum of those two elements, which the algorithm will return and terminate.\\
Inductive Hypothesis: assume that the algorithm is correct for all inputs of size $1 \le k < n$.\\
Let $X$ have the median, $m1$, and let $Y$ have the median, $m2$. Let the length of $X + Y = n$.\\
If $m1 = m2$ then there will $n/2$ elements less than or equal to $m1$ and $n/2$ elements greater than or equal to $m1$.\\
Similarly, there will $n/2$ elements less than or equal to $m2$ and $n/2$ elements greater than or equal to $m2$.\\
There will be $n$ elements less than or equal to $m1=m2$ and $n$ elements greater than or equal to $m1=m2$, so the median of both arrays will be $m1=m2$. The algorithm will then return $m1$ and terminate.\\
If $m1 > m2$, then the median of the two arrays must have a value, $a$, such that $m2 < a < m1$.\\
Then $a$ must exist in either $X[0..i],~X[i]=m1$ or $Y[j..n/2],~Y[j]=m2$.\\
The algorithm will then recursively call the algorithm on these two arrays of size $n/2$, which will be correct by the inductive hypothesis.\\
If $m1 < m2$, then the median of the two arrays must have a value, $a$, such that $m1 < a < m2$.\\
Then $a$ must exist in either $X[i..n/2],~X[i]=m1$ or $Y[0..j],~Y[j]=m2$.\\
The algorithm will then recursively call the algorithm on these two arrays of size $n/2$, which will be correct by the inductive hypothesis.\\
Therefor, the algorithm will always return the correct median and terminate. $\Box$\\
\item
The run-time of the algorithm can be given by $T(n) = T(\ceil{n/2}) + c$.\\
There are only two recursive function calls in the algorithm, and only one of these will ever be called. The rest of the algorithm has constant run time.\\
\item
Using case 2 of the Master theorem on $T(n) = T(\ceil{n/2}) + c$, we find that $a=1$, $b=2$, $log_ba=log_21=0$, $f(n)=c=\Theta (n^{log_ba=0})$.\\
Thus $T(n) = \Theta (lg~n)$.\\
\end{enumerate}
\end{answer}

\end{problemlist}
\end{document}
