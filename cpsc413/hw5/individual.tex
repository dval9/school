\documentclass{assignment}

\coursetitle{Design and Analysis of Algorithms 1}
\courselabel{CPSC 413}
\exercisesheet{Home Work \#5}{individual version}
\student{Tom Crowfoot - 10037477}
\semester{Winter 2014}
\usepackage{amsmath}
\usepackage{amssymb}
\usepackage{amsthm}
\usepackage[linesnumbered,noline]{algorithm2e}
\usepackage{mathtools}
\DeclarePairedDelimiter\floor{\lfloor}{\rfloor}
\DeclarePairedDelimiter\ceil{\lceil}{\rceil}

\begin{document}
\begin{problemlist}

\pbitem
\begin{problem}
\end{problem}
\begin{answer}
\\
\begin{enumerate}
\item
\begin{align*}
T(n)=&T(\floor{n/2})+T(\floor{n/4})+T(\floor{n/8})+n\\
=&T(n/2)+T(n/4)+T(n/8)+n\\
=&T(7n/8)+n\\
=&T(7^2n/8^2)+7n/8+n\\
=&T(7^3n/8^3)+7^2n/8^2+7n/8+n\\
\end{align*}
So a guess is $n\sum_{i=0}\frac{7}{8}^i=n\frac{1}{1-7/8}\in \Theta (n)$\\
\item
Claim: $T(n)=T(\floor{n/2})+T(\floor{n/4})+T(\floor{n/8})+n\le c*n$\\
Base case: If $n<1000$, then $T(n)=a \le c*n=c*1000$\\
So pick $a=c$.\\
IH: $T(k)=T(\floor{k/2})+T(\floor{k/4})+T(\floor{k/8})+k\le a*k$ for $k < n$\\
\begin{align*}
T(n)=&T(\floor{n/2})+T(\floor{n/4})+T(\floor{n/8})+n\\
\le&a*n/2+a*n/4+a*n/8+n\\
=&a*15n/8\\
\le&c*n\\
\Rightarrow&a*15n=c*8n\\
\end{align*}
So pick $2a=c$, and then $T(n)=O(n)$.\\

Base case: If $n<1000$, then $T(n)=a \ge c*n=c*1000$\\
So pick $a=c*1000$.\\
IH: $T(k)=T(\floor{k/2})+T(\floor{k/4})+T(\floor{k/8})+k\ge a*k$ for $k\le n$\\
\begin{align*}
T(n)=&T(\floor{n/2})+T(\floor{n/4})+T(\floor{n/8})+n\\
\ge&a*n/2+a*n/4+a*n/8+n\\
=&a*15n/8\\
\ge&c*n\\
\Rightarrow&a*15n=c*8n\\
\end{align*}
So pick $a=c$, and then $T(n)=\Omega (n)$.\\

Thus $T(n) = \Theta (n)$.\\
\end{enumerate}
\end{answer}

\clearpage
\pbitem
\begin{problem}
\end{problem}
\begin{answer}
\\
\begin{enumerate}
\item
$T(n)=2T(n/2)+n^3$, $a=2$, $b=2$, $log_22=1$, $f(n)=n^3=\Omega (n^{1+\epsilon})$, $2(n^3/8)\le cn^3$, $c=1/4$.\\
Thus $T(n) = \Theta(f(n)) = \Theta(n^3)$.\\ 
\item
$T(n)=2T(n/2)+n$, $a=2$, $b=2$, $log_22=1$, $f(n)=n=\Theta (n^1)$.\\
Thus $T(n) = \Theta(nlogn)$.\\
\item
$T(n)=2T(n/2)+\sqrt(n)$, $a=2$, $b=2$, $log_22=1$, $f(n)=\sqrt(n)=O(n^{1-\epsilon})$.\\
Thus $T(n)=\Theta(n)$.\\
\item
$T(n)=4T(n/8)+\sqrt{n}log^2n$, $a=4$, $b=8$, $log_84=2/3$, $f(n)=\sqrt{n}log^2n=O(n^{2/3}-\epsilon)$.\\
Thus $T(n)=\Theta (n^{2/3})$.\\
\item
$T(n)=16T(n/7)+n^2$, $a=16$, $b=7$, $f(n)=n^2=\Omega (n^{log_7 16 + \epsilon})$, $16(n^2/49)\le cn^2$, $k=16/49$.\\
Thus $T(n)=\Theta (n^2)$.\\
\end{enumerate}
\end{answer}

\clearpage
\pbitem
\begin{problem}
\end{problem}
\begin{answer}
\\
\begin{enumerate}
\item
Strassen's Algorithm can be modified by first taking the $n\times n$ matrix, say A, and extending it's size to a power of two, filling in the new spots with zeros. That is if A is $n\times n$, where $n \neq 2^i$ for some integer i, then $2^{k-1} \le n \le 2^k$ and we can create a new matrix B of size $2^k\times 2^k$, where the new matrix has elements $b_{ij}=a_{ij}$ for $i=[0..n],j=[0..n]$ and $b_{ij}=0$ for $i=[n+1..k],j=[n+1..k]$. Having done this for both matrices you wish to multiply, apply the algorithm, and then extract the $n\times n$ elements from the result to get the desired result.\\
\item
The modification to the algorithm is to increase the input size from $n$ to $k$. Then the new runtime of the algorithm will be $\Theta (k^{lg 7})$. However $k<2n$, so we get $\Theta ((2n)^{lg 7})=\Theta (2^{lg 7}n^{lg 7})=\Theta (n^{lg 7})$.\\
\end{enumerate}
\end{answer}

\clearpage
\pbitem
\begin{problem}
\end{problem}
\begin{answer}
\\
\begin{enumerate}
\item
Pre-conditions: Two arrays X \& Y, length(X)=length(Y)=n, all elements in X \& Y distinct, X \& Y sorted ascending order.\\
Post-conditions: The median of the two arrays, x.\\
\item
Find the median of two sorted arrays.\\
\IncMargin{3em}
\begin{algorithm}
  int median(X, Y)\\\{\\
  \Indp
  m1 = X[n/2]\;
  m2 = Y[n/2]\;
  \uIf{$length~X~==~1~and~length~Y~==~1$}{
    \Return{min(X[0], Y[0])}
  }
  \uElseIf{$m1 == m2$}{
    \Return{m1}
  }
  \uElseIf{$m1 > m2$}{
    \Return{median(X[0..n/2], Y[n/2..n])}
  }
  \uElseIf{$m1 < m2$}{
    \Return{median(X[n/2..n], Y[0..n/2])}
  }
  \Indm
  \}
\end{algorithm}
\DecMargin{3em}
\item
Base case: $n=1$\\
If $n=1$, both arrays consist of only one element.\\
The algorithm will return the minimum of these two, which will be the median.\\
Base case: $m1=m2$\\
If the median of the first array equals the median of the second array, we found the median, and the algorithm returns.\\
IH: assume the algorithm is correct for all input size $k < n$.\\
Let X have median $m1$, Y have median $m2$, and length of X and Y be $n$. If $m1=m2$ the algorithm will return the correct median. If $m1 > m2$, the algorithm will call itself on input size $n/2$, so that will return the correct median by assumption. If $m1 < m2$, the algorithm will call itself on input size $n/2$, so that will return the correct median by assumption. Thus the algorithm will always return the correct solution, and terminate. $\Box$\\
\item
All linex except 11 and 14 are constant time operations. Line 11 calls the function recursively on an input of size $n/2$, line 14 also calls the function recursively on an input of size $n/2$, however only one of these will ever be called. We then get the recurrence relation:\\
$T(n)=T(n/2) + c$\\
\item
Using the Master Theorem on $T(n)=T(n/2) + c$, we get $a=1$, $b=2$, $log_21=0$, $f(n)=c=\Theta (n^0)$.\\
Thus $T(n)=\Theta (n^0lg n)=\Theta (lg n)$.\\
\end{enumerate}
\end{answer}

\end{problemlist}
\end{document}
