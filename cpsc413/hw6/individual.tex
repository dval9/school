\documentclass{assignment}

\coursetitle{Design and Analysis of Algorithms 1}
\courselabel{CPSC 413}
\exercisesheet{Home Work \#6}{Individual version}
\student{Tom Crowfoot - 10037477}
\semester{Winter 2014}
\usepackage{amsmath}
\usepackage{amssymb}
\usepackage{amsthm}
\usepackage[linesnumbered,noline]{algorithm2e}
\usepackage{mathtools}
\DeclarePairedDelimiter\floor{\lfloor}{\rfloor}
\DeclarePairedDelimiter\ceil{\lceil}{\rceil}

\begin{document}
\begin{center}
\renewcommand{\arraystretch}{2}
\begin{tabular}{|c|c|c|} \hline
Problem & Marks \\ \hline \hline
1 & \\ \hline
2 & \\ \hline
Total & \\ \hline
\end{tabular}
\end{center}

\bigskip

\begin{problemlist}

\clearpage
\pbitem
\begin{problem}
\end{problem}
\begin{answer}
\\
\begin{enumerate}
\item
Consider the jobs $H=\{50,0,0\}$ and $L=\{10,10,10\}$. Then $h_2 < l_1 + l_2$, so it will choose $l_1$, then $h_3 < l_2 + l_3$, so it will choose $l_2$, finally $h_4 < l_3 + l_4$, so it will choose $l_3$. Then the maximum value of $\{l_1,l_2,l_3\}$ is $30$. However it is clear the best result would be to choose $h_1,l_2,l_3$, to get a maximum of $70$.
\item
Pre-conditions: an array H which contains the profits for completing high-stress jobs, length H =  n; an array L which contains the profits for completing low-stress jobs, length L = n.\\
Post-conditions: the maximum value that can be earned, x.
\item
Sub-problem: find the maximum profit for up to week $i$. An expression for the maximum value of week $i$ would be $bestjobs(i) = max(h_i + bestjobs(i-2), l_i + bestjobs(i-1))$ for $i>2$, $bestjobs(2) = max(h_2, l_2 + bestjobs(i-1))$, and $bestjobs(1) = max(h_1, l_1)$.
\item
Recursive algorithm to compute maximum value:\\
\IncMargin{3em}
\begin{algorithm}
  int maxRec(H, L, n)\\\{\\
  \Indp
  \uIf{$n==1$}{\Return{$max(h_1,l_1)$}\;}
  \uIf{$n==2$}{\Return{$max(h_2, l_2+maxRec(H,L,n-1))$}\;}
  \Return{$max(h_n + maxRec(H,L,n-2), l_n + maxRec(H,L,n-1))$\;}
  \Indm
  \}
\end{algorithm}
\DecMargin{3em}
\item
$T(n) = c + T(n-2) + T(n-1)$, for $n>2$.
\item
$T(6) = c + T(4) + T(5)$\\
$T(6) = 3c + T(2) + T(3) + T(3) + T(4)$\\
$T(6) = 7c + T(1) + T(2) + T(1) + T(2) + T(2) + T(3)$\\
$T(6) = 13c + T(1) + T(2)$\\
$T(6) = 15c$\\
A guess is that $T(n)\in O(2^n)$.\\
Claim: $T(n) = c + T(n-2) + T(n-1) \le a*2^n$.\\
Base case: If $n \le 2$, then $T(n) = c \le a*2^n$.\\
We can then pick $a=c$, and this is true.\\
Inductive Hypothesis: $T(k) = c + T(k-2) + T(k-1) \le a*2^k$, for $k<n$.\\
\begin{align*}
T(n) =& c + T(n-2) + T(n-1)\\
\le&c + a*2^{n-2} + a*2^{n-1}\\
\le&c + a*2^n*2^{-2} + a*2^n*2^{-1}\\
\le&c + a*2^n(2^{-2} + 2^{-1})\\
\le&c + \frac{3a}{4}2^n\\
\le&a*2^n\\
\end{align*}
If we choose $a=1$, then we find that $T(n)\in O(2^n)$.
\clearpage
\item
Dynamic algorithm to compute maximum value, and the choices made:\\
\IncMargin{3em}
\begin{algorithm}
  int maxDyn(H, L, n)\\\{\\
  \Indp
  P[n]\;
  J[n]\;
  $a_1 = h_1$\;
  $a_2 = l_2$\;
  P[1] = max($a_1, a_2$)\;
  \uIf{$a_1 > a_2$}{$J[1] = 1$\;}
  \uElse{$J[1] = 0$\;}
  $a_1 = h_2$\;
  $a_2 = l_2 + P[1]$\;
  P[2] = max($a_1, a_2$)\;
  \uIf{$a_1 > a_2$}{$J[2] = 1$\;}
  \uElse{$J[2] = 0$\;}
  \For{$i=3~to~n$}{
    $a_1 = h_i + P[i-2]$\;
    $a_2 = l_i + P[i-1]$\;
    P[i] = max($a_1, a_2$)\;
    \uIf{$a_1 > a_2$}{$J[i] = 1$\;}
    \uElse{$J[i] = 0$\;}
  }
  print(Maximum profit=P[n])\;
  \For{$i=n~to~1$}{
    \uIf{$J[i] == 1$}{print(Take high-stress job in week $i$, Take no job in week $i-1$)\;$i-=2$\;}
    \uElse{print(Take low-stress job in week $i$)\;$i-=1$\;}
  }
  \Indm
  \}
\end{algorithm}
\DecMargin{3em}
\item
The algorithm consists of a two for loops doing constant time work, each of size $n$.
The runtime of the algorithm is then $c + 2*\sum_{i=1}^n(c)$.\\
We can then use the limit test to find the tight bound:
\begin{align*}
&c + 2*\sum_{i=1}^n(c)\\
=&c + 2*c*n\\
&\lim_{n\rightarrow \infty} \frac{c + 2*c*n}{n}\\
=&\lim_{n\rightarrow \infty} \frac{c/n + 2*c}{1}\\
=&2*c\\
\end{align*}
As the limit resolves to a constant, the runtime is $\Theta (n)$.\\
\end{enumerate}
\end{answer}

\clearpage
\pbitem
\begin{problem}
\end{problem}
\begin{answer}
\\
\begin{enumerate}
\item
Pre-conditions: An array p[n][n] representing a checker board that contains the cost for moving to that square.\\
Post-conditions: A path of length n, starting at the bottom of the checker board going to the top, that maximizes the dollars gathered, of value d.
\item
$d[i,j] = max(d[i-1,j-1]+p[i][j], d[i-1,j]+p[i][j], d[i-1,j+1]+p[i][j])$, where $i$ is the row, and $j$ is the column. Out of bound elements have a value of $0$.
\item
Recursive algorithm to compute maximum value for some square:\\
\IncMargin{3em}
\begin{algorithm}
  int maxRec(p, i, j)\\\{\\
  \Indp
  \uIf{$i\le 0 || i>n || j\le 0 || j>n$}{\Return{$0$\;}}
  \uIf{$i==1$}{\Return{$p[i][j]$\;}}
  \Else{\Return{$max(maxRec(p,i-1,j-1)+p[i][j], maxRec(p,i-1,j)+p[i][j],maxRec(p,i-1,j+1)+p[i][j])$}\;}
  \Indm
  \}
\end{algorithm}
\DecMargin{3em}
Driver program to compute maximal value:\\
\IncMargin{3em}
\begin{algorithm}
  int maxRecDriver(p, i, j)\\\{\\
  \Indp
  $d=-\infty$\;
  \For{$j=1~to~n$}{
    $max=maxRec(p, n, j)$\;
    \uIf{$max > d$}{$d = max$\;}
  }
  print(Max value = d)\;
  \Indm
  \}
\end{algorithm}
\DecMargin{3em}
\item
$T(n) = c + 3T(n-1)$.
\item
$T(6) = c + 3T(5)$\\
$T(6) = 4c + 9T(4)$\\
$T(6) = 13c + 27T(3)$\\
$T(6) = 40c + 81T(2)$\\
$T(6) = 121c + 243T(1)$\\
$T(6) = 364c$\\
A guess is that $T(n)\in O(3^n)$.\\
Claim: $T(n) = c + 3T(n-1) \le a*3^n$.\\
Base case: If $n = 1$, then $T(n) = c \le a*3^n$.\\
We can then pick $a=c$, and this is true.\\
Inductive Hypothesis: $T(k) = c + 3T(k-1) \le a*3^k$, for $k<n$.\\
\begin{align*}
T(n) =& c + 3T(n-1)\\
\le&c + 3*a*3^{n-1}\\
\le&c + a*3^n\\
\le&a*3^n\\
\end{align*}
If we choose $a=1$, then we find that $T(n)\in O(3^n)$.
\clearpage
\item
Dynamic algorithm to compute maximum value, and the choices made:\\
\IncMargin{3em}
\begin{algorithm}
  int maxDyn(p, n)\\\{\\
  \Indp
  d[n][n], c[n][n]\;
  \For{$j=1~to~n$}{
    $d[1][j]=p[1][j]$\;
  }
  \For{$i=2~to~n$}{
    \For{$j=1~to~n$}{
      $a_1=d[i-1][j-1]+p[i][j]$\;
      $a_2=d[i-1][j]+p[i][j]$\;
      $a_3=d[i-1][j+1]+p[i][j]$\;
      $d[i][j]=max(a_1, a_2, a_3)$\;
      \uIf{$a_1 \ge a_2 \&\& a_1 \ge a_3$}{$c[i][j]=0$\;}
      \uIf{$a_2 \ge a_1 \&\& a_2 \ge a_3$}{$c[i][j]=1$\;}
      \uIf{$a_3 \ge a_1 \&\& a_3 \ge a_2$}{$c[i][j]=2$\;}
    }
  }
  $d = -\infty$, $c = -1$\;
  \For{$j=1~to~n$}{
    \uIf{$d[n][j] > d$}{$d = d[n][j]$\;$c=j$\;}
  }
  print(Max profit to be made is d)\;
  \For{$i=n~to~1$}{
    print(Choose square i,c)\;
    \uIf{$c[i][c]==0$}{$c-=1$\;}
    \uIf{$c[i][c]==1$}{$c=c$\;}
    \uIf{$c[i][c]==2$}{$c+=1$\;}
  }
  \Indm
  \}
\end{algorithm}
\DecMargin{3em}
\item
The algorithm consists of two for loops of size $n$ doing constant work, and one double for loop from with one size $n-1$ and the other $n$ doing constant work.\\
Thus the runtime of the algorithm is $c + 2*\sum_{i=1}^n(c) + \sum_{j=1}^{n-1}\sum_{k=1}^{n}(c)$.\\
We can then use the limit test to find the tight bound:
\begin{align*}
&c + 2*\sum_{i=1}^n(c) + \sum_{j=1}^{n-1}\sum_{k=1}^n(c)\\
=&c + 2*c*n + \sum_{j=1}^{n-1}(c*n)\\
=&c + 2*c*n + c*n*(n-1)\\
=&c + c*n^2\\
&\lim_{n\rightarrow \infty} \frac{c + c*n^2}{n^2}\\
=&\lim_{n\rightarrow \infty} \frac{c/n^2 + c}{1}\\
=&c\\
\end{align*}
As the limit resolves to a constant, the runtime is $\Theta (n^2)$.\\
\end{enumerate}
\end{answer}

\end{problemlist}
\end{document}

