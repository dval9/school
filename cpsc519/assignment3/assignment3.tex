\documentclass{assignment}
\usepackage{amsmath,amssymb,amsfonts}

\coursetitle{Quantum Algorithms}
\courselabel{CPSC 519}
\exercisesheet{Homework \#1}{}
\student{Tom Crowfoot - 10037477}
\semester{Winter 2015}

\newcommand{\inner}[2]{\ensuremath{\langle{#1}|{#2}\rangle}}

\input{Qcircuit}

\begin{document}
\begin{problemlist}

  \pbitem
  \begin{problem}
  \end{problem}
  \begin{answer}
    \\
    \begin{enumerate}
    \item
      To calculate the parity of a function $f$ classically, you need to calculate the value of $f$ on its entire domain, and sum the values. Then the algorithm is $\sum_{i=0}^{N-1}f(i)$ $mod$ $2$.\\
      This algorithm will require $N$ calls to the oracle.
    \item
      Again to compute the parity of a function, you need to calculate the value of the function on each value and sum the results. From Deutch's algorithm, we know that we can compute a function at two values simultaneously with a quantum computer. We can then compute the value of the function quantumly, with $N/2$ queries to the oracle.\\
      This algorithm has to be optimal. The parity of $f(x)$ is essentially random, so no information can be gained about the parity of $f(x+1)$ from $f(x)$. Thus the function must be evaluated over the entire domain. Then the best that can be done is two evaluations per query of the oracle, and so there must be $N/2$ queries.\\
    \end{enumerate}
  \end{answer}

  \pbitem
  \begin{problem}
  \end{problem}
  \begin{answer}
    \\
    \begin{enumerate}
    \item
      \begin{align*}
      \ket{1}\ket{-}=&\ket{10}-\ket{11}\\
      U_f\ket{1}\ket{-}=&U_f\ket{10}-U_f\ket{11}\\
      =&\ket{1}\ket{0\oplus f(1)}-\ket{1}\ket{1\oplus f(1)}\\
      =&\ket{1}(\ket{0\oplus f(1)}-\ket{1\oplus f(1)})\\
      =&(-1)^{f(1)}\ket{1}\ket{-}\\
      \end{align*}
    \item
      \begin{align*}
        \ket{\phi}=&(S\otimes 1)U_f\frac{1}{\sqrt{3}}(\ket{0}+\ket{1}+\ket{2})\ket{-}\\
        =&(S\otimes 1)\frac{1}{\sqrt{3}}(U_f\ket{0}\ket{-}+U_f\ket{1}\ket{-}+U_f\ket{2}\ket{-})\\
        =&(S\otimes 1)\frac{1}{\sqrt{3}}((-1)^{f(0)}\ket{0}\ket{-}+(-1)^{f(1)}\ket{1}\ket{-}+(-1)^{f(2)}\ket{2}\ket{-})\\
        =&(S\otimes 1)\frac{1}{\sqrt{3}}((-1)^{f(0)}\ket{0}\ket{-}+(-1)^{f(1)}\ket{1}\ket{-}+(-1)^{f(2)}\ket{2}\ket{-})\\
        =&\frac{1}{\sqrt{3}}((-1)^{f(0)}\ket{0}\ket{-}+(-1)^{f(1)}\ket{1}\ket{-}-(-1)^{f(2)}\ket{2}\ket{-})\\
        =&\frac{1}{\sqrt{3}}((-1)^{f(0)}\ket{0}\ket{-}+(-1)^{f(1)}\ket{1}\ket{-}-\ket{2}\ket{-})\\
        =&\frac{1}{\sqrt{3}}((-1)^{f(0)}\ket{0}+(-1)^{f(1)}\ket{1}-\ket{2})\ket{-}\\
      \end{align*}
    \item
      \begin{align*}
        |\inner{b_1}{\phi}|=&|\frac{1}{\sqrt{3}}(\bra{0}+\bra{1}+\bra{2})\frac{1}{\sqrt{3}}((-1)^{f(0)}\ket{0}+(-1)^{f(1)}\ket{1}-\ket{2})|\\
        =&|\frac{1}{\sqrt{3}}(\bra{0}+\bra{1}+\bra{2})\frac{1}{\sqrt{3}}(-1\ket{0}+\ket{1}-\ket{2})|\\
        =&|\frac{1}{3}(-\inner{0}{0}+\inner{1}{1}-\inner{2}{2})|\\
        =&|\frac{1}{3}(-1)|\\
        =&|\frac{-1}{3}|\\
        =&\frac{1}{3}\\
      \end{align*}
    \item
      From the answer in part 3, the probability to measure state $\ket{b_1}$ is
      $|\inner{b_1}{\phi}|^2=|\frac{1}{3}|^2=\frac{1}{9}$.\\
    \item
      $f(0)=0,f(1)=0$\\
      \begin{align*}
        |\inner{b_1}{\phi}|^2=&|\frac{1}{\sqrt{3}}(\bra{0}+\bra{1}+\bra{2})\frac{1}{\sqrt{3}}((-1)^{f(0)}\ket{0}+(-1)^{f(1)}\ket{1}-\ket{2})|^2\\
        =&|\frac{1}{\sqrt{3}}(\bra{0}+\bra{1}+\bra{2})\frac{1}{\sqrt{3}}(\ket{0}+\ket{1}-\ket{2})|^2\\
        =&|\frac{1}{3}(1)|^2\\
        =&\frac{1}{9}\\
      \end{align*}
      $f(0)=0,f(1)=1$\\
      \begin{align*}
        |\inner{b_1}{\phi}|^2=&|\frac{1}{\sqrt{3}}(\bra{0}+\bra{1}+\bra{2})\frac{1}{\sqrt{3}}((-1)^{f(0)}\ket{0}+(-1)^{f(1)}\ket{1}-\ket{2})|\\
        =&|\frac{1}{\sqrt{3}}(\bra{0}+\bra{1}+\bra{2})\frac{1}{\sqrt{3}}(\ket{0}-\ket{1}-\ket{2})|^2\\
        =&|\frac{1}{3}(-1)|^2\\
        =&\frac{1}{9}\\
      \end{align*}
      $f(0)=1,f(1)=1$\\
      \begin{align*}
        |\inner{b_1}{\phi}|^2=&|\frac{1}{\sqrt{3}}(\bra{0}+\bra{1}+\bra{2})\frac{1}{\sqrt{3}}((-1)^{f(0)}\ket{0}+(-1)^{f(1)}\ket{1}-\ket{2})|^2\\
        =&|\frac{1}{\sqrt{3}}(\bra{0}+\bra{1}+\bra{2})\frac{1}{\sqrt{3}}(-\ket{0}-\ket{1}-\ket{2})|^2\\
        =&|\frac{1}{3}(-3)|^2\\
        =&1\\
      \end{align*}
    \item
      If at the end of the algorithm if you do not measure $\ket{b_1}$, then return $f(0)\land f(1)=0$.\\
      Otherwise, if you measure $\ket{b_1}$, output $1$.\\
      The chance to measure $\ket{b_1}$ while $f(0)\land f(1) \neq 1$ is $\frac{1}{9}+ \frac{1}{9}+ \frac{1}{9} = \frac{1}{3}$.\\
      Then the probability of success for the algorithm is $1-\frac{1}{3}=\frac{2}{3}$.\\
    \end{enumerate}
  \end{answer}

  \clearpage
  \pbitem
  \begin{problem}
  \end{problem}
  \begin{answer}
    \\
    \begin{enumerate}
    \item
      \begin{align*}
        P(F_2\otimes F_3)Q
        =&P(
        \begin{bmatrix}
          1 & 1\\
          1 & \omega_2
        \end{bmatrix}
        \otimes
        \begin{bmatrix}
          1 & 1 & 1\\
          1 & \omega_3 & \omega_3^2\\
          1 & \omega_3^2 & \omega_3
        \end{bmatrix}
        )Q\\
        =&P(
        \begin{bmatrix}
          1 & 1 & 1 & 1 & 1 & 1\\
          1 & \omega_6^2 & \omega_6^4 & 1 & \omega_6^2 & \omega_6^4\\
          1 & \omega_6^4 & \omega_6^2 & 1 & \omega_6^4 & \omega_6^2\\
          1 & 1 & 1 & \omega_6^3 & \omega_6^3 & \omega_6^3\\
          1 & \omega_6^2 & \omega_6^4 & \omega_6^3 & \omega_6^5 & \omega_6\\
          1 & \omega_6^4 & \omega_6^2 & \omega_6^3 & \omega_6 & \omega_6^5\\
        \end{bmatrix}
        )Q\\
        =&
        \begin{bmatrix}
          1 & 1 & 1 & 1 & 1 & 1\\
          1 & \omega_6^1 & \omega_6^2 & \omega_6^3 & \omega_6^4 & \omega_6^5\\
          1 & \omega_6^2 & \omega_6^4 & 1 & \omega_6^2 & \omega_6^4\\
          1 & \omega_6^3 & 1 & \omega_6^3 & 1 & \omega_6^3\\
          1 & \omega_6^4 & \omega_6^2 & 1 & \omega_6^4 & \omega_6^2\\
          1 & \omega_6^5 & \omega_6^4 & \omega_6^3 & \omega_6^2 & \omega_6\\
        \end{bmatrix}\\
        =&F_6
      \end{align*}
      Then $P$ will be the matrix to permute the rows, and $Q$ will permute the columns.\\
      We want P to permute such that row $1\rightarrow 1, 2\rightarrow 5, 3\rightarrow 3, 4\rightarrow 4, 5\rightarrow 2, 6\rightarrow 6$.\\
      Then \begin{equation*}P=
      \begin{bmatrix}
        1 & 0 & 0 & 0 & 0 & 0\\
        0 & 0 & 0 & 0 & 1 & 0\\
        0 & 0 & 1 & 0 & 0 & 0\\
        0 & 0 & 0 & 1 & 0 & 0\\
        0 & 1 & 0 & 0 & 0 & 0\\
        0 & 0 & 0 & 0 & 0 & 1
      \end{bmatrix}
      \end{equation*}
      We want Q to permute such that column $1\rightarrow 1, 2\rightarrow 3, 3\rightarrow 5, 4\rightarrow 4, 5\rightarrow 6, 6\rightarrow 2$.\\
      Then \begin{equation*}Q=
      \begin{bmatrix}
        1 & 0 & 0 & 0 & 0 & 0\\
        0 & 0 & 1 & 0 & 0 & 0\\
        0 & 0 & 0 & 0 & 1 & 0\\
        0 & 0 & 0 & 1 & 0 & 0\\
        0 & 0 & 0 & 0 & 0 & 1\\
        0 & 1 & 0 & 0 & 0 & 0
      \end{bmatrix}
      \end{equation*}
    \item
      \begin{align*}
        P(F_2\otimes F_3)Q
        =&P(
        \begin{bmatrix}
          1 & 1\\
          1 & \omega_2
        \end{bmatrix}
        \otimes
        \begin{bmatrix}
          1 & 1\\
          1 & \omega_2
        \end{bmatrix}
        )Q\\
        =&P(
        \begin{bmatrix}
          1 & 1 & 1 & 1\\
          1 & \omega_4^2 & 1 & \omega_4^2\\
          1 & 1 & \omega_4^2 & \omega_4^2\\
          1 & \omega_4^2 & \omega_4^2 & 1\\
        \end{bmatrix}
        )Q\\
        =&\begin{bmatrix}
          1 & 1 & 1 & 1\\
          1 & \omega_4 & \omega_4^2 & \omega_4^3\\
          1 & \omega_4^2 & 1 & \omega_4^2\\
          1 & \omega_4^3 & \omega_4^2 & \omega_4\\
        \end{bmatrix}\\
        =&F_4        
      \end{align*}
      On inspection of the matrix $F_2\otimes F_2$, it lacks a term with $\omega_4^3$ in it. As $F_4$ does have this term, it will be impossible to permute $(F_2\otimes F_2)$ into $F_4$.\\
    \end{enumerate}
  \end{answer}

  \pbitem
  \begin{problem}
  \end{problem}
  \begin{answer}
    \\
    \begin{enumerate}
    \item
      $controlled-S_N\ket{s}\ket{t}=\omega_N^{st}\ket{s}\ket{t}$\\
      If $\ket{s}$ is 0, then the $S_N$ should not be applied to $\ket{t}$. The expression then gives $controlled-S_N\ket{0}\ket{t}=\omega_N^{0*t}\ket{0}\ket{t}=\ket{0}\ket{t}$ as required.\\
      If $\ket{s}$ is 1, then the $S_N$ should be applied to $\ket{t}$. The expression then gives $controlled-S_N\ket{1}\ket{t}=\omega_N^{1*t}\ket{1}\ket{t}=\omega_N^t\ket{1}\ket{t}$.\\
      If $\ket{s}$ is larger than 1, the expression will give $\omega_N^{s*t}\ket{s}\ket{t}$.\\ 
    \item
      First note that
      \begin{align*}
        &controlled-T_N(\frac{1}{N}\sum_{x=0,y=0}^{N-1}\omega_N^{xs+(y-x)t}\ket{x}\ket{y-x\text{ modulo }N})\\
        =&\sum_{x=0,y=0}^{N-1}\omega_N^{x(s-t)+yt}\ket{x}\ket{y}\\
        =&\ket{\phi_N^{s-t}}\ket{\phi_N^t}
      \end{align*}
      We can then compute
      \begin{align*}
        controlled-T_N\ket{\phi_N^s}\ket{\phi_N^t}=&controlled-T_N\sum_{x=0,y=0}^{N-1}\omega_N^{x(s-t)+y't}\ket{x}\ket{y'}\\
        =&\sum_{x=0,y=0}^{N-1}\omega_N^{x(s-t)+y't}\ket{x}\ket{y'+x\text{ modulo }N}\\
        =&\sum_{x=0,y=0}^{N-1}\omega_N^{x(s-t)+y''t}\ket{x}\ket{y''}\\
        =&\ket{\phi_N^{s-t}}\ket{\phi_N^{t}}
      \end{align*}
    \end{enumerate}
  \end{answer}
  
\end{problemlist}
\end{document}
