\documentclass{assignment}
\usepackage{amsmath, amssymb, amsfonts}

\coursetitle{Cryptography}
\courselabel{PMAT 429}
\exercisesheet{Home Work \#1}{}
\student{Tom Crowfoot - 10037477}
\semester{Winter 2015}

\begin{document}

\begin{problemlist}
\pbitem
\begin{problem}
\end{problem}
\begin{answer}
  \\
  \begin{enumerate}
  \item f
  \item e
  \item g
  \item i
  \item a
  \item a
  \item a
  \item g
  \end{enumerate}
\end{answer}

\pbitem
\begin{problem}
\end{problem}
\begin{answer}
  \\
  \begin{enumerate}
  \item $n^3 + 14n - 20 \le cn^3$, for all $n \ge n_0$.\\
    Setting $n_0=3$ and $c=2$ gives
    \begin{align*}
      0 &\le cn^3 - n^3- 14n + 20\\
      &= 2*3^3 - 3^3- 14*3 + 20\\
      &= 27 - 42 + 20\\
      &= 5
    \end{align*}
    Then $n^3 + 14n - 20 \in O(n^3)$. $\Box$
    \clearpage
  \item $$\lim_{n\to \infty} \frac{n^3 + 2 ln(n)}{n^3}$$.\\
    Using L'Hôpital's rule to simplify yields
    \begin{align*}
      &\lim_{n\rightarrow \infty} \frac{3n^2+2/n}{3n^2}\\
      \Rightarrow &\lim_{n\to \infty} \frac{3 + 2/n^2}{3}\\
      = &1
    \end{align*}
    As the limit goes to $1$, $n^3 + 2ln(n) \sim n^3$. $\Box$
  \end{enumerate}
\end{answer}


\pbitem
\begin{problem}
\end{problem}
\begin{answer}
  \\
  Let $EA(a,b)=n$ be the number of steps taken by the Euclidean Algorithm on inputs $a,b$.\\
  By Lamé's Theorem, $b \ge F_{n+1} \ge \alpha^{n-1}$, where $F_{n+1}$ is the $n+1$ Fibonacci number, and $\alpha$ is the golden ratio. Also, $log_{10}(\alpha) > \frac{1}{5}$.\\
  \begin{align*}
    b &\ge \alpha^{n-1}\\
    log_{10}(b) &\ge log_{10}(\alpha^{n-1})\\
    log_{10}(b) &> (n-1)\frac{1}{5}\\
    5*log_{10}(b) + 1 &> n\\
    5*log_{10}(b) &\ge n
  \end{align*}
  Then the number of divisions needed is not more than five times the number of decimal digits in the smaller of the two integers. $\Box$
\end{answer}

\clearpage
\pbitem
\begin{problem}
\end{problem}
\begin{answer}
  \\
  \begin{enumerate}
  \item
    Step 1:\\
    $g = 1, u=27, v=37$\\
    Step 2:\\
    $u \not\equiv v \not\equiv 0$, procede to step 3\\
    Step 3:\\
    $t = |27 - 37|/2 = 5$\\
    $u = 27, v = 5$\\
    $t = |27 - 5| / 2 = 11$\\
    $u = 11, v = 5$\\
    $t = |11-5|/2 = 3$\\
    $u = 3, t = 5$\\
    $t = |3-5|/2 = 1$\\
    $u = 3, v = 1$\\
    $t = |3-1|/2 = 1$\\
    $u = 1, v = 1$\\
    $t = |1-1|/2 = 0$\\
    $u = 0, v = 1$\\
    Step 4:\\
    Return $1 * 1$;
  \item
    Steps 1 and 4 have $O(1)$ runtime.\\
    Step 2 in the worst case will require $O(log(a))$ operations to shift, assuming that $a \ge b$. Shift operation takes constant time, and can be preformed maximum number of bits on the largest number before causing both to be zero. This step will never be executed in worst cast however, as step 3 is far more costly.\\
    Worst case for step 3 will have lines i and ii never executed. Worst case will have the value of $u$ large and $v$ small, so it will iterate $O(log(a))$ times, halving $u$ each iteration. The subtraction in the loop costs $O(max(log(a),log(b))$ to calculate. The cost of the loop is $log(n)^2$, where $n =max(a,b)$.\\
    The total cost of the algorithm is then $O(log(n)^2) + O(log(a)) + O(1) + O(1)=O(log(n)^2)$.\\
  \end{enumerate}
\end{answer}

%\clearpage
\pbitem
\begin{problem}
\end{problem}
\begin{answer}
  \\
Let $G$ be a finite group, then $\# G = lcm(ord(g_1),ord(g_2),\ldots,ord(g_n))$.\\
By Lagrange's Theorem, for $g \in G$, $ord(g) \bigm| |G|$.\\
As $ord(g) \bigm| \# G$, it follows that $\#G \bigm| |G|$.
\end{answer}


\pbitem
\begin{problem}
\end{problem}
\begin{answer}
  \begin{enumerate}
  \item Assume there is another irreducible polynomial in $\mathbb{F}_2[x]$ of degree less than or equal to 2. Then it must have the form $ax^2 + bx + c$, where $a,b,c \in \{0,1\}$.\\
    The only remaining possibilities are:\\
    $x^2 = x*x$\\
    $x^2 + x = x * (x + 1)$\\
    $x^2 + 1 \Rightarrow 1^2 +1 = 0$.\\
    Then there cannot be another irreducible polynomial in $\mathbb{F}_2[x]$.
  \item Assume that $x^4 +x +1$ is reducible in $\mathbb{F}_2[x]$, so it can be factored into irreducables in $\mathbb{F}_2[x]$.\\
    The possibilities are:
    $(x^2 + x + 1) * (x^2 + x + 1) = x^4 + x^2 + 1$\\
    $(x^2 + x + 1) * (x + 1) * (x + 1) = x^4 + x ^ 3 + x + 1$\\
    $(x^2 + x + 1) * (x + 1) * (x) = x^4 + x$\\
    $(x^2 + x + 1) * (x) * (x) = x^4 + x ^ 3 + x^2$\\
    $(x + 1) * (x + 1) * (x + 1) * (x + 1) = x^4 + 1$\\
    $(x + 1) * (x + 1) * (x + 1) * (x) =x^4+x^3+x^2+x$\\
    $(x + 1) * (x + 1) * (x) * (x) = x^4 + x^2$\\
    $(x + 1) * (x) * (x) * (x) =x^4 + x^3$\\
    $(x) * (x) * (x) * (x) = x^4$\\
    Then as $x^4 +x +1$ cannot be factored, it must be irreducable.
  \item $(x^3 + x + 1)(x^3 + x^2) = x^6 + x^5 + x^4 + x^2$
    \begin{align*}
      x^6 + x^5 + x^4 + x^2 &= x^2(x^4) + x(x^4) + x^4 + x^2\\
      &= x^2(x + 1) + x(x + 1) + x + 1 + x^2\\
      &= x^3 + x^2 + x^2 + x + x + 1 + x^2\\
      &= x^3 + x^2 + 1
    \end{align*}
  \item $(x^2 + x + 1)^{-1} = (x^2 + x + 1)^{2^4 - 2}$
    \begin{align*}
      (x^2 + x + 1)^{2^4 - 2} &= (x^2 + x + 1)^{14}\\
      &= x^28 + x^26 + x^22 + x^20 + x^16 + x^14 + x^12 + x^8 + x^6 + x^2 + 1\\
      &= (x^4)^7 + x^2(x^4)^6 + x^2(x^4)^5 + (x^4)^5 + (x^4)^4 + x^2(x^4)^3 + (x^4)^3 + (x^4)^2 + x^2(x^4) + x^2 + 1\\
      &= (x + 1)^7 + x^2(x + 1)^6 + x^2(x + 1)^5 + (x + 1)^5 + (x + 1)^4 + x^2(x + 1)^3 + (x + 1)^3 + (x + 1)^2 + x^2(x + 1) + x^2 + 1\\
      &= x^8 + x^6 + x^5 + x^4 + x^3 + x\\
      &= (x^4)^2 + x^2(x^4) + x(x^4) + x^4 + x^3 + x\\
      &= (x + 1)^2 + x^2(x+1) + x(x+1) + x + 1 + x^3 + x\\
      &= x^2 + 1 + x^3 + x^2 + x^2 + x + x + 1 + x^3 + x\\
      &= x^2 + x\\
    \end{align*}
  \item The possible orders of $x$ are $1, 3, 5, 15$.\\
    $x \neq 1$, and $x^3 \neq 1$.\\
    $x^4 = x + 1$, so $x^5 = x(x^4) = x(x+1)=x^2 + x \neq 1$\\
    Thus $ord(x) = 15$, so it is primitive.
  \end{enumerate}
\end{answer}

\end{problemlist}
\end{document}
