\documentclass{assignment}
\usepackage{amsmath, amssymb, amsfonts}
\usepackage[lined,boxed,linesnumbered]{algorithm2e}

\coursetitle{Cryptography}
\courselabel{PMAT 429}
\exercisesheet{Home Work \#3}{}
\student{Tom Crowfoot - 10037477}
\semester{Winter 2015}

\begin{document}

\begin{problemlist}
\pbitem
\begin{problem}
\end{problem}
\begin{answer}
  \\
  \begin{align*}
    x=&n-\phi(n)\\
    =&pq-(p-1)(q-1)\\
    =&pq-pq+p+q-1\\
    =&p+q-1\\
    =&p+n/p-1\\
    0=&p+n/p-1-x\\
    =&p^2+n-p-px\\
    =&p^2+p(-x-1)+n\\
  \end{align*}
  You can then use the quadratic formula to solve for $p$.
\end{answer}
\clearpage
\pbitem
\begin{problem}
\end{problem}
\begin{answer}
  \\
  We first note that if $n$ is a perfect power, then $k \le \lfloor log(n)\rfloor$.\\
  We can compute an approximation of $a$ from $n^{1/k}$ for all $k$, starting at $k=2$.\\
  If for some $k$ we get $a^k=n$ we are done and can output $a,k$.\\
  If $k=\lfloor log(n)\rfloor$, and $a^k\neq n$, then we can halt and output FALSE.\\
  To compute the values of $a$, we can use a value of $x_{max}=2^{\lfloor log(n)/k\rfloor +1}$ as an upper bound, and then binary search on the range of $2$ to $x_{max}$.\\
  To compute the value $a^k$ we can use binary exponentation.\\
  \begin{algorithm}
    \For{$k=2; k\le log$ $n;k++$}{
      $a =\lfloor n^{1/k} \rfloor$\;
      \If{$n==a^k$}{
        \Return{$a,k$}\;
      }
    }
    \Return{FALSE}\;
  \end{algorithm}
  \\
  From the above, the algorithm is clearly correct.\\
  In computing the k-th root values for $a$, binary search will iterate $O((log$ $n)/k)$ times.\\
  For each iteration of the binary search, the guess of the value is raised to the k-th power. Using an efficient powering algorithm will take $O((log$ $n)^2)$ time to compute this.\\
  Then the total cost of computing the k-th root will be $O(((log$ $n)^3)/k)$.\\
  Summing this cost over all of the values of $k$ gives $\sum_{k=2}^{log(n)} \frac{1}{k} (log$ $n)^3$.\\
  We can use the fact that $\sum_{k=2}^{log(n)} \frac{1}{k} \le log(log(n))$ to simplify.\\
  Then the total cost of our algorithm is $O((log$ $n)^3log(log(n)))$.\\
  Clearly $log(log(n) \le log(n)$, so $(log$ $n)^3log(log(n)) \le (log$ $n)^4$.\\
  Then the algorithm is in $O(log$ $n)^5$.\\
  The best bound for an algorithm to compute perfect powers is $O((log$ $n)^{1+O(1)})$ according to some paper online that I found.
\end{answer}

\pbitem
\begin{problem}
\end{problem}
\begin{answer}
  \\
  \begin{tabular}{c|c|c|c}
    q&$k_q$&$q^{k_q}$&a mod n\\
    \hline
    2 & 14 & 16384 & 20401\\
    3 & 9 & 19683 & 12553\\
    5 & 6 & 15625 & 13440\\
  \end{tabular}
  \\Then $gcd(13440-1,26123) = 151$, so $n=151*173$.\\
\end{answer}

\pbitem
\begin{problem}
\end{problem}
\begin{answer}
  \\
  \begin{tabular}{c|c|c|c}
    i&$x_i$&$x_{2i}$&d\\
    \hline
    1 & 5 & 26 & 1\\
    2 & 26 & 458330 & 1\\
    3 & 677 & 1283597 & 1\\
    4 & 458330 & 762979 & 1\\
    5 & 580868 & 588992 & 677\\
  \end{tabular}
  \\Then one of the factors of $n$ is $677$, and the other is $n/677=2017$.\\
\end{answer}

\end{problemlist}
\end{document}
