\documentclass{assignment}
\usepackage{amsmath, amssymb, amsfonts}
\usepackage{algorithmic,algorithm}

\coursetitle{Cryptography}
\courselabel{PMAT 429}
\exercisesheet{Take Home Final}{}
\student{Tom Crowfoot - 10037477}
\semester{Winter 2015}

\begin{document}

\begin{problemlist}
\pbitem
\begin{problem}
  Recalling the DHP: given $p,\beta, \beta^x, \beta^y$, find $\beta^{xy}$.\\
  Additionally, EGP: given $p, \beta, \beta^x, \beta^y, \beta^{xy}\cdot m$, find $m$.\\
  To see that $DHP \equiv_P EGP$, we must show $DHP \le_P EGP$ and $EGP \le_P DHP$.\\
  \begin{enumerate}
  \item $EGP \le_P DHP$.\\
    Assume that we have an oracle solving the $DHP$.\\
    We are given inputs $p, \beta, \alpha=\beta^x, \gamma_1=\beta^y, \gamma_2=\beta^{xy}\cdot m$, and we wish to find $m$.\\
    We can use something like the EEA to find inverses in polynomial time, so it is easy to compute $\alpha^{-1}=\beta^{-x}$.\\
    We can then pass inputs $p, \beta, \alpha^{-1}, \gamma_1$ to our $DHP$ solving oracle. The oracle will then return the value $\beta^{-xy}$.\\
    We can then compute the value $\beta^{-xy}\gamma_2=\beta^{-xy}\beta^{xy}\cdot m=m$, solving the $EGP$.
  \item $DHP \le_P EGP$.\\
    Assume that we have an oracle solving the $EGP$.\\
    We are given inputs $p, \beta, \beta^x, \beta^y$, and we wish to find $\beta^{xy}$.\\
    We can then pass inputs $p, \beta, \beta^x, \beta^y, 1$ to our $EGP$ solving oracle. The oracle will then return the value $m=\gamma_1^{-x}\gamma_2$, where $\gamma_1=\beta^y$ and $\gamma_2=1$, so $m=\beta^{-xy}$.\\
    We can then find the inverse of $\beta^{-xy}$ in polynomial time using EEA, to get $\beta^{xy}$, solving the $DHP$.
  \end{enumerate}
  Then since $DHP \le_p EGP$ and $EGP \le_p DHP$, $EGP \equiv_p DHP$ as required. $\blacksquare$
\end{problem}

\pbitem
\begin{problem}
  \begin{enumerate}
  \item
    If we can solve the DLP with base $b$, then we can solve the DLP with base $\beta_i$.\\
    As $b$ is a generator of $G$, there must be some $z_i$ such that $b^{z_i}=\beta_i$.\\
    Then to solve $\alpha_i=\beta_i^{x_i}$, we first solve $\beta_i=b^{z_i}$ to get $z_i$.\\
    Then $\alpha_i=(b^{z_i})^{x_i}=b^{z_i*x_i}$, so we can solve to get $z_i*x_i$.\\
    Finally we can compute $\frac{z_i*x_i}{z_i}=x_i$ to solve this DLP.\\
    
    So to solve for $x_i$, we needed to solve two DLP's with base $b$.\\
    Then to solve for $t$ DLP's to find all the $x_i$, we need to solve $2t$ DLP's with base $b$.\\
  \item
    Using the idea from the previous part, we could solve $b^{z_i}=\beta_i$ and $\alpha_i=b^{z_i*x_i}$ to compute $x_i$.\\
    The normal BSGS algorithm runs in $\sqrt{n}$ time, where most of the time is spent computing the table of values to compare against. If we can increase the table size to $\sqrt{tn}$, we can decrease the amount of computations to find collisions with the table to $\sqrt{n}/t$ for each $x_i$.\\

    \begin{algorithm}[H]
      \begin{algorithmic}[1]
        \STATE Set $m=\lceil \sqrt{tn} \rceil$.
        \FOR{$0\le j < m$}
        \STATE{Store $(j,(b^j)$ in hash table $A$.}
        \ENDFOR
        \STATE Compute $b^{-m}$.
        \FOR{$1\le k \le t$}
        \FOR{$0\le i < m/t$}
        \IF{$\alpha_i*b^{-im}$ in table $A$.}
        \STATE{$z_i*x_i=im+j$, so store $(i,im+j)$ in hash table $B$.}
        \ENDIF
        \IF{$\beta_i*b^{-im}$ in table $A$.}
        \STATE{$z_i=im+j$, so store $(i,im+j)$ in hash table $C$.}
        \ENDIF
        \ENDFOR
        \ENDFOR
        \FOR{$1\le i < t$}
        \STATE{$x_i=(z_i*x_i)/z_i$}
        \ENDFOR
        \end{algorithmic}
    \end{algorithm}

    Line $1$ is calculating the square root of an integer, which can be done fast.\\
    The loop on line $3$ iterates $m$ times, and each iteration preforms a single multiplication, so it preforms $m$ multiplications.\\
    Line $5$ is a single inversion, which can be done with EEA, so this is fast.\\
    The loop on line $6$ iterates $t$ times.\\
    The inner loop on line $7$ iterates $m/t$ times.\\
    Lines $8$ and $11$ both preform a single multiplication.\\
    So the inner loop preforms a total of $2*m/t$ multiplications.\\
    Then the outer loop preforms a total of $t*2*m/t=2*m$ multiplications.\\
    The loop on line 16 preforms $t$ divisions, which we can count the same as multiplications.\\

    Then the total run time of the algorithm will be $O(m+2m+t)=O(m)=O(\sqrt{tn})$ multiplications.\\
    
    
  \end{enumerate}
\end{problem}

\pbitem
\begin{problem}
  \begin{enumerate}
  \item
    To show this, we can just compute:\\
    $780^3\pmod{1333} \equiv 1\pmod{1333}$\\
    $1296^3\pmod{1333} \equiv 1\pmod{1333}$
  \item
    From part $(a)$, we know that $780$ and $1296$ both have order $3$.\\
    Then note that if $a^3 \equiv 1\pmod{p}$ and $b^3 \equiv 1\pmod{p}$, then $(a*b)^3 \equiv 1\pmod{p}$
    So we can compute:
    \begin{align*}
      780*780 \equiv 552\pmod{1333}\\
      780*1296 \equiv 466\pmod{1333}\\
      780*466 \equiv 904\pmod{1333}\\
      1296*1296 \equiv 36\pmod{1333}\\
      1296*466 \equiv 87\pmod{1333}\\
      1296*904 \equiv 1210\pmod{1333}\\
    \end{align*}
    Then $780,1296,552,466,904,36,87,1210$ all have order $3$.
  \item
    To factor $n=1333$, we can compute the gcd of $n$ and the difference between two of these numbers.\\
    Then $gcd(1333, 780-1296) = 43$, so $1333=43*31$.\\
    Additional pairs that work are: $(780,904), (780,36), (780,1210), (1296,552), (1296,87),$ $(1296,1210), (552,466), (552,36), (552,87), (466,36), (466,1210), (904,36), (904,87)$.\\
  \end{enumerate}
\end{problem}

\pbitem
\begin{problem}
  \begin{enumerate}
  \item
    First, $341-1 = 340 = 2^2*85$. If $341$ is a strong pseudoprime to base $a$, then either $a^{85} \equiv 1 \pmod{341}$ or $a^{85*2^s}\equiv -1\pmod{341}$ for $s=0,1$.\\
    Then for base $2$, $2^{85} \equiv 32 \pmod{341}$, $2^{85*2} \equiv 1 \pmod{341}$.\\
    So $341$ is not a strong pseudoprime to the base $2$.
  \item
    From the previous, we know $2^{85*2}=32^2 \equiv 1 \pmod{341}$.\\
    Rewriting, we get $32^2-1\equiv 0 \pmod{341}$, where $32^2-1=(32-1)(32+1)=31*33\equiv 0 \pmod{341}$.\\
    So either $31$ or $33$ divides $341$, and by checking both choices we find $341=31*11$.
  \item
    If $n$ is a pseudoprime to the base $a$, then $a^{n-1}\equiv 1 \pmod{n}$.\\
    However $n$ is not a strong pseudoprime to the base $a$, so first write $n-1=2^x*y$. Then $a^y \not\equiv 1 \pmod{n}$ and $a^{y*2^s}\not\equiv -1 \pmod{n}$ for $s=0,1,2,...,x-1$.\\
    So $a^{y*2^x} \equiv 1\pmod{n}$, and then $a^{y*2^{x-1}}*a^{y*2^{x-1}} \equiv 1 \pmod{n}$.\\
    For simplicity write $y*2^{x-1} = z$, and then $a^{2z}-1 \equiv 0 \pmod{n}$.\\
    Then $(a^z+1)(a^z-1)\equiv 0 \pmod{n}$, so either one of $a^z-1$ or $a^z+1$ divides $n$.\\
    However we know that $a^z \not\equiv -1 \pmod{n}$ so $a^z+1 \not\equiv 0 \pmod{n}$, so $a^z-1$ must be a non-trivial divisor of $n$. $\blacksquare$    
  \end{enumerate}
\end{problem}

\pbitem
\begin{problem}
  The largest diverse number is $9876543210$, so the largest prime diverse number must be smaller than this.\\
  Ideally we could take a permutation of this number and get a prime, as the more digits the number has the larger it is.\\
  We can't use all digits as the sum of digits would then be $45$ which is divisible by $3$ and so the number is divisible by $3$.\\
  So if we remove one of $1,2,4,5,7,8$ it won't be divisible by $3$.\\
  However we need this number to end in $1,3,7,9$ if we want it to possibly be prime.\\
  This leaves us with $4*(8!-7!)=141120$ remaining numbers that could be the largest diverse prime.\\
  This is quite a small number for a computer to check, so we can write a small program to quickly check and find the largest diverse prime.\\
  I attached the java program I wrote to do this, it runs in roughly $5$ seconds on my laptop, so pretty fast.\\
  Then $987654103$ is the largest prime diverse number.
\end{problem}

\pbitem
\begin{problem}
  From the four points, we get four linear equations
  \begin{align*}
    120^2=&96^3+96a+b\pmod{p}\\
    17^2=&118^3+118a+b\pmod{p}\\
    89^2=&73^3+73a+b\pmod{p}\\
    121^2=&97^3+97a+b\pmod{p}\\
  \end{align*}
  We also know that $4a^3+27b^2\neq 0$.\\
  Also, $p > 121$.\\

  We can use matrix row reductions to find values for $a$ and $b$ satisfying two equations.\\
  From the first two equations, we get $a=-772407/22$ and $b=27501840/11$.\\
  From the first and third, we get $a=-489240/23$ and $b=26949312/23$.\\
  From the first and fourth, we get $a=-27696$ and $b=1788480$.\\
  From the second and third, we get $a=-140183/5$ and $b=8327879/5$.\\
  From the second and fourth, we get $a=-248237/7$ and $b=17792765/7$.\\
  From the third and fourth, we get $a=-21539$ and $b=1191251$.\\
  
  Equating all these things for $a$, we get\\
  $-772407/22+489240/23-(-27696+140183/5)-(-248237/7+21539)=0 \pmod{p}$, so
  \[
  -4520941/17710=-4520941=0 \pmod{p}
  \].
  Doing the same for $b$, we get\\
  $27501840/11-26949312/23-1788480+8327879/5-17792765/7+1191251=0 \pmod{p}$, so
  \[
  -1284140731/8855=-1284140731=0 \pmod{p}
  \].
  Then both $-4520941$ and $-1284140731$ have the prime factor $-131$, so maybe we should try $p=131$.\\

  Then the equations become
  \begin{align*}
    28=&96a+b\pmod{131}\\
    128=&118a+b\pmod{131}\\
    114=&73a+b\pmod{131}\\
    104=&97a+b\pmod{131}\\
  \end{align*}
  Equating these, we can try to find a value for $a$.\\
  Then we get $28-96a-128+118a-114+73a+104-97a=0 \pmod{131}$, so
  \[
  -2a -110 =0 \pmod{131}
  \]
  So maybe $a=76$.

  Then finding the value of $b$ is easy, we can solve
  \[
  b=28-96*76 \pmod{131}
  \]
  So we get $b=68$.\\

  Now checking the original equations with our found values
  \begin{align*}
    120^2=&96^3+96(76)+68\pmod{131}\\
    \Rightarrow121=&121\pmod{131}\\
    17^2=&118^3+118(76)+68\pmod{131}\\
    \Rightarrow27=&27\pmod{131}\\
    89^2=&73^3+73(76)+68\pmod{131}\\
    \Rightarrow61=&61\pmod{131}\\
    121^2=&97^3+97(76)+68\pmod{131}\\
    \Rightarrow100=&100\pmod{131}\\
  \end{align*}
  and everything seems to work.\\
  So then the field must be $\mathbb{F}_{131}$, and $a=76$, $b=68$.

  If we are only given $3$ points, we can try the same method.\\
  If we were given only the first three points in this question(but any three should still work), when we equate the equations to get values for $a$ and $b$ using only the relations between the first and second, and first and third equations, we can again find that both numbers have prime divisor $131$, and so again guess $p=131$.\\
  Going on, we can try to find $a$ using only the first two now reduced equations, and find $a=76$ again.\\
  Finding $b$ can be done with just the first equation.\\
  Then we have still found the field, $a$, and $b$ with just three points given.
\end{problem} 

\end{problemlist}
\end{document}
