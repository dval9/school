\documentclass[letterpaper]{article}
\usepackage{amsmath,amsthm,amsfonts}
\usepackage[margin=1in]{geometry}
\usepackage{algorithmic}
\usepackage{algorithm}
\usepackage{mathtools}
\usepackage{booktabs}
\usepackage{float}

\DeclarePairedDelimiter{\bra}{\langle}{\rvert}
\DeclarePairedDelimiter{\ket}{\lvert}{\rangle}
\DeclarePairedDelimiter{\set}{\{}{\}}
\DeclarePairedDelimiter{\abs}{\lvert}{\rvert}
\DeclarePairedDelimiter{\paren}{(}{)}
\DeclarePairedDelimiter{\floor}{\lfloor}{\rfloor}
\DeclareMathOperator{\ord}{ord}
\newcommand{\op}[1]{\textsf{#1}}
\newcommand{\oracle}{\op{U}_f}
\newcommand{\complex}{\mathbb{C}}
\newcommand{\tensor}{\otimes}
\newcommand{\ints}{\mathbb{Z}}
\renewcommand{\o}[2]{\omega_{#1}^{#2}}
\setlength{\parindent}{0pt}
\setlength{\parskip}{1em}

\title{Shor's Algorithm for Factoring and DLP}
\author{Arvin Sonuhi, Kevin Chum, Tom Crowfoot}
\date{}

%    Q-circuit version 2
%    Copyright (C) 2004  Steve Flammia & Bryan Eastin
%    Last modified on: 9/16/2011
%
%    This program is free software; you can redistribute it and/or modify
%    it under the terms of the GNU General Public License as published by
%    the Free Software Foundation; either version 2 of the License, or
%    (at your option) any later version.
%
%    This program is distributed in the hope that it will be useful,
%    but WITHOUT ANY WARRANTY; without even the implied warranty of
%    MERCHANTABILITY or FITNESS FOR A PARTICULAR PURPOSE.  See the
%    GNU General Public License for more details.
%
%    You should have received a copy of the GNU General Public License
%    along with this program; if not, write to the Free Software
%    Foundation, Inc., 59 Temple Place, Suite 330, Boston, MA  02111-1307  USA

% Thanks to the Xy-pic guys, Kristoffer H Rose, Ross Moore, and Daniel Müllner,
% for their help in making Qcircuit work with Xy-pic version 3.8.  
% Thanks also to Dave Clader, Andrew Childs, Rafael Possignolo, Tyson Williams,
% Sergio Boixo, Cris Moore, Jonas Anderson, and Stephan Mertens for helping us test 
% and/or develop the new version.

\usepackage{xy}
\xyoption{matrix}
\xyoption{frame}
\xyoption{arrow}
\xyoption{arc}

\usepackage{ifpdf}
\ifpdf
\else
\PackageWarningNoLine{Qcircuit}{Qcircuit is loading in Postscript mode.  The Xy-pic options ps and dvips will be loaded.  If you wish to use other Postscript drivers for Xy-pic, you must modify the code in Qcircuit.tex}
%    The following options load the drivers most commonly required to
%    get proper Postscript output from Xy-pic.  Should these fail to work,
%    try replacing the following two lines with some of the other options
%    given in the Xy-pic reference manual.
\xyoption{ps}
\xyoption{dvips}
\fi

% The following resets Xy-pic matrix alignment to the pre-3.8 default, as
% required by Qcircuit.
\entrymodifiers={!C\entrybox}

\newcommand{\bra}[1]{{\left\langle{#1}\right\vert}}
\newcommand{\ket}[1]{{\left\vert{#1}\right\rangle}}
    % Defines Dirac notation. %7/5/07 added extra braces so that the commands will work in subscripts.
\newcommand{\qw}[1][-1]{\ar @{-} [0,#1]}
    % Defines a wire that connects horizontally.  By default it connects to the object on the left of the current object.
    % WARNING: Wire commands must appear after the gate in any given entry.
\newcommand{\qwx}[1][-1]{\ar @{-} [#1,0]}
    % Defines a wire that connects vertically.  By default it connects to the object above the current object.
    % WARNING: Wire commands must appear after the gate in any given entry.
\newcommand{\cw}[1][-1]{\ar @{=} [0,#1]}
    % Defines a classical wire that connects horizontally.  By default it connects to the object on the left of the current object.
    % WARNING: Wire commands must appear after the gate in any given entry.
\newcommand{\cwx}[1][-1]{\ar @{=} [#1,0]}
    % Defines a classical wire that connects vertically.  By default it connects to the object above the current object.
    % WARNING: Wire commands must appear after the gate in any given entry.
\newcommand{\gate}[1]{*+<.6em>{#1} \POS ="i","i"+UR;"i"+UL **\dir{-};"i"+DL **\dir{-};"i"+DR **\dir{-};"i"+UR **\dir{-},"i" \qw}
    % Boxes the argument, making a gate.
\newcommand{\meter}{*=<1.8em,1.4em>{\xy ="j","j"-<.778em,.322em>;{"j"+<.778em,-.322em> \ellipse ur,_{}},"j"-<0em,.4em>;p+<.5em,.9em> **\dir{-},"j"+<2.2em,2.2em>*{},"j"-<2.2em,2.2em>*{} \endxy} \POS ="i","i"+UR;"i"+UL **\dir{-};"i"+DL **\dir{-};"i"+DR **\dir{-};"i"+UR **\dir{-},"i" \qw}
    % Inserts a measurement meter.
    % In case you're wondering, the constants .778em and .322em specify
    % one quarter of a circle with radius 1.1em.
    % The points added at + and - <2.2em,2.2em> are there to strech the
    % canvas, ensuring that the size is unaffected by erratic spacing issues
    % with the arc.
\newcommand{\measure}[1]{*+[F-:<.9em>]{#1} \qw}
    % Inserts a measurement bubble with user defined text.
\newcommand{\measuretab}[1]{*{\xy*+<.6em>{#1}="e";"e"+UL;"e"+UR **\dir{-};"e"+DR **\dir{-};"e"+DL **\dir{-};"e"+LC-<.5em,0em> **\dir{-};"e"+UL **\dir{-} \endxy} \qw}
    % Inserts a measurement tab with user defined text.
\newcommand{\measureD}[1]{*{\xy*+=<0em,.1em>{#1}="e";"e"+UR+<0em,.25em>;"e"+UL+<-.5em,.25em> **\dir{-};"e"+DL+<-.5em,-.25em> **\dir{-};"e"+DR+<0em,-.25em> **\dir{-};{"e"+UR+<0em,.25em>\ellipse^{}};"e"+C:,+(0,1)*{} \endxy} \qw}
    % Inserts a D-shaped measurement gate with user defined text.
\newcommand{\multimeasure}[2]{*+<1em,.9em>{\hphantom{#2}} \qw \POS[0,0].[#1,0];p !C *{#2},p \drop\frm<.9em>{-}}
    % Draws a multiple qubit measurement bubble starting at the current position and spanning #1 additional gates below.
    % #2 gives the label for the gate.
    % You must use an argument of the same width as #2 in \ghost for the wires to connect properly on the lower lines.
\newcommand{\multimeasureD}[2]{*+<1em,.9em>{\hphantom{#2}} \POS [0,0]="i",[0,0].[#1,0]="e",!C *{#2},"e"+UR-<.8em,0em>;"e"+UL **\dir{-};"e"+DL **\dir{-};"e"+DR+<-.8em,0em> **\dir{-};{"e"+DR+<0em,.8em>\ellipse^{}};"e"+UR+<0em,-.8em> **\dir{-};{"e"+UR-<.8em,0em>\ellipse^{}},"i" \qw}
    % Draws a multiple qubit D-shaped measurement gate starting at the current position and spanning #1 additional gates below.
    % #2 gives the label for the gate.
    % You must use an argument of the same width as #2 in \ghost for the wires to connect properly on the lower lines.
\newcommand{\control}{*!<0em,.025em>-=-<.2em>{\bullet}}
    % Inserts an unconnected control.
\newcommand{\controlo}{*+<.01em>{\xy -<.095em>*\xycircle<.19em>{} \endxy}}
    % Inserts a unconnected control-on-0.
\newcommand{\ctrl}[1]{\control \qwx[#1] \qw}
    % Inserts a control and connects it to the object #1 wires below.
\newcommand{\ctrlo}[1]{\controlo \qwx[#1] \qw}
    % Inserts a control-on-0 and connects it to the object #1 wires below.
\newcommand{\targ}{*+<.02em,.02em>{\xy ="i","i"-<.39em,0em>;"i"+<.39em,0em> **\dir{-}, "i"-<0em,.39em>;"i"+<0em,.39em> **\dir{-},"i"*\xycircle<.4em>{} \endxy} \qw}
    % Inserts a CNOT target.
\newcommand{\qswap}{*=<0em>{\times} \qw}
    % Inserts half a swap gate.
    % Must be connected to the other swap with \qwx.
\newcommand{\multigate}[2]{*+<1em,.9em>{\hphantom{#2}} \POS [0,0]="i",[0,0].[#1,0]="e",!C *{#2},"e"+UR;"e"+UL **\dir{-};"e"+DL **\dir{-};"e"+DR **\dir{-};"e"+UR **\dir{-},"i" \qw}
    % Draws a multiple qubit gate starting at the current position and spanning #1 additional gates below.
    % #2 gives the label for the gate.
    % You must use an argument of the same width as #2 in \ghost for the wires to connect properly on the lower lines.
\newcommand{\ghost}[1]{*+<1em,.9em>{\hphantom{#1}} \qw}
    % Leaves space for \multigate on wires other than the one on which \multigate appears.  Without this command wires will cross your gate.
    % #1 should match the second argument in the corresponding \multigate.
\newcommand{\push}[1]{*{#1}}
    % Inserts #1, overriding the default that causes entries to have zero size.  This command takes the place of a gate.
    % Like a gate, it must precede any wire commands.
    % \push is useful for forcing columns apart.
    % NOTE: It might be useful to know that a gate is about 1.3 times the height of its contents.  I.e. \gate{M} is 1.3em tall.
    % WARNING: \push must appear before any wire commands and may not appear in an entry with a gate or label.
\newcommand{\gategroup}[6]{\POS"#1,#2"."#3,#2"."#1,#4"."#3,#4"!C*+<#5>\frm{#6}}
    % Constructs a box or bracket enclosing the square block spanning rows #1-#3 and columns=#2-#4.
    % The block is given a margin #5/2, so #5 should be a valid length.
    % #6 can take the following arguments -- or . or _\} or ^\} or \{ or \} or _) or ^) or ( or ) where the first two options yield dashed and
    % dotted boxes respectively, and the last eight options yield bottom, top, left, and right braces of the curly or normal variety.  See the Xy-pic reference manual for more options.
    % \gategroup can appear at the end of any gate entry, but it's good form to pick either the last entry or one of the corner gates.
    % BUG: \gategroup uses the four corner gates to determine the size of the bounding box.  Other gates may stick out of that box.  See \prop.

\newcommand{\rstick}[1]{*!L!<-.5em,0em>=<0em>{#1}}
    % Centers the left side of #1 in the cell.  Intended for lining up wire labels.  Note that non-gates have default size zero.
\newcommand{\lstick}[1]{*!R!<.5em,0em>=<0em>{#1}}
    % Centers the right side of #1 in the cell.  Intended for lining up wire labels.  Note that non-gates have default size zero.
\newcommand{\ustick}[1]{*!D!<0em,-.5em>=<0em>{#1}}
    % Centers the bottom of #1 in the cell.  Intended for lining up wire labels.  Note that non-gates have default size zero.
\newcommand{\dstick}[1]{*!U!<0em,.5em>=<0em>{#1}}
    % Centers the top of #1 in the cell.  Intended for lining up wire labels.  Note that non-gates have default size zero.
\newcommand{\Qcircuit}{\xymatrix @*=<0em>}
    % Defines \Qcircuit as an \xymatrix with entries of default size 0em.
\newcommand{\link}[2]{\ar @{-} [#1,#2]}
    % Draws a wire or connecting line to the element #1 rows down and #2 columns forward.
\newcommand{\pureghost}[1]{*+<1em,.9em>{\hphantom{#1}}}
    % Same as \ghost except it omits the wire leading to the left. 


\begin{document}
	\maketitle{}
	\section{Introduction to Quantum Computing}
	\subsection{Dirac Notation}
	In classical computing we use two discrete states called \emph{bits}, namely
	0 and 1. For quantum computing, we can be in some value between 0 and 1. The
	base states for a quantum computer are usually denoted $\ket0$ and $\ket1$
	(``ket''-0 and ``ket''-1). These have the forms:
	\begin{align*}
		\ket0 = \begin{bmatrix}1\\0\end{bmatrix} &&\text{and} && \ket1 =
		\begin{bmatrix}0\\1\end{bmatrix}.
	\end{align*}
	We can extend these to higher dimensions, and has a natural notation
	involving \emph{tensor products}:
	\begin{align*}\hspace{-0.5em}
		\ket0 = \ket{00} = \ket0\tensor\ket0 \begin{bmatrix}1\\0\\0\\0\end{bmatrix}, &&
		\ket1 = \ket{01} = \ket0\tensor\ket1 \begin{bmatrix}0\\1\\0\\0\end{bmatrix}, &&
		\ket2 = \ket{10} = \ket1\tensor\ket0 \begin{bmatrix}0\\0\\1\\0\end{bmatrix}, &&
		\ket3 = \ket{11} = \ket1\tensor\ket1 \begin{bmatrix}0\\0\\0\\1\end{bmatrix}
	\end{align*}
	and so on. The latter states are in $\complex^{2}\tensor\complex^{2}$, while
	the former are in $\complex^2$. So we can generalize to dimensions that may
	not be a power of two if needed.

	States can also be in a superposition of any basis state, for example the
	state $\alpha\ket0+\beta\ket1$, where $\alpha,\beta\in\complex$ and
	$\alpha^2+\beta^2=1$. Note that any quantum state must have a unit norm.
	
	Dirac notation offers other benefits, but these are the only properties we need.
	
	At times we will want to ``measure'' a given state in terms of the basis states. Since a general state may be in a superposition of some combination of these states, we do not always see any given basis state with equal probability. For example, when measuring $\alpha\ket0+\beta\ket1$, we see $\ket0$ with probability $\abs{\alpha}^2$ and we see $\ket1$ with probability $\abs{\beta}^2$.
	\subsection{Gates and the Quantum Fourier Transform}
	Operators in quantum computing are represented by unitary
	matrices---matrices whose inverses are their transpose complex conjugates
	(this is so that operations are reversible). The most important operation
	that is used in the algorithms we discuss is the Quantum Fourier Transform
	(QFT) mod $N$:
		\begin{align*}
			\op{F}_N = \frac{1}{\sqrt{N}}
			\begin{bmatrix}1 & 1 & 1 & \cdots & 1\\
				1 & \o{N}{} & \o{N}{2} & \cdots & \o{N}{N-1}\\
				1 & \o{N}{2} & \o{N}{3} & \cdots & \o{N}{2(N-1)}\\
				1 & \o{N}{3} & \o{N}{6} & \cdots & \o{N}{3(N-1)}\\
				\vdots & \vdots & \vdots &  \ddots & \vdots\\
				1 & \o{N}{N-1} & \o{N}{2(N-1)} & \cdots &
				\o{N}{(N-1)(N-1)}
		\end{bmatrix},
		\end{align*}
	where the $\o{N}{}$ is the $N^{th}$ root of unity
	$\o{N}{}=\exp(\frac{2\pi i}{N})$.

	\textbf{Remark.} $\sum_{j=0}^{N-1}\o{N}{j} = \o{N}{1} + \o{N}{2} + \cdots +
	\o{N}{N-1} = \frac{1-\o{N}{N}}{1-\o{N}{}} = 0$, since $\o{N}{N}=1$.

	For a given state $\ket{x}$, we have \[\op{F}_N\ket{x} =
	\frac{1}{\sqrt{N}}\sum_{i=0}^{N-1} \o{N}{xi}\ket{i} := \ket{\phi^x_N}\] and
	\[\op{F}_N^{-1}\ket{\phi^x_N} = \op{F}_N^{\dagger}\ket{\phi^x_N} = \ket{x}\]
	\subsection{Circuits}
	Just as classical computers can be built with \texttt{NAND} gates alone, we
	could (in theory) build quantum computers with three gates: \op{H} (Hadamard
			gate), 
	\op{CNOT} (controlled-not), and $\op{R}_{\pi/4}$ (phase shift). The details
	of which are unimportant for this discussion. Using these gates we can build the QFT gate as
	described in the previous section.

	We draw quantum circuits like normal circuits, with one row for each quantum
	register, and gates drawn in boxes. Registers go from top-to-bottom, so the bottom row is the least significant qubit. Circuits are read left-to-right. When we write the results as equations, operators are  applied starting from the \emph{right}.
	\section{Shor's Algorithm for Factoring}
	\subsection{The Algorithm}
	\begin{algorithm}[H]
		\caption{Classical part of Shor's algorithm}
		\texttt{INPUT:} Integer $n$ a pseudoprime to be factored

		\texttt{OUTPUT:} Nontrivial factor of $n$
		\begin{algorithmic}[1]
			\STATE Randomly pick $a \in \set{3,\dots,n-1}$.
			\STATE Compute $d=\gcd(a,n)$
			\IF{$d\neq 1$}
				\RETURN $d$
			\ENDIF
			\STATE Compute $r = \ord(a)$ (Quantum part)
			\IF{$r$ is even \textbf{and} $\gcd(a^{r/2},n)\neq -1$}
				\STATE Compute $x=\gcd(a^{r/2}\pm1,n)$
				\IF{$x\neq 1$ and $x\neq n$}
					\RETURN $x$
				\ENDIF
			\ENDIF
			\RETURN ``failure''
		\end{algorithmic}
	\end{algorithm}
	\subsection{Analysis of the Classical Part}
		\subsubsection{Correctness}
		Note that $a^r\equiv 1 \pmod{n}$ by Fermat's Little Theorem. This means
		that $a^r-1\equiv 0\pmod{n}$. So $n\mid a^r-1$. If $r$ is an even
		number, then $n\mid (a^{r/2}-1)(a^{r/2}+1)$. So any factor of $n$ is
		also a factor of either $a^{r/2}-1$ or $a^{r/2}+1$.  Assume we found $r$
		as an order of $a$. If $r$ were odd, then computing $a^{r/2}$ is
		impossible as 2 does not divide $r$. If $a^{r/2}\equiv-1\pmod{n}$, then
		$a^{r/2}+1\equiv 0\pmod{n}$. So when $\gcd(a^{r/2}+1,n)$ is not trivial,
		it must be a factor of $n$.


		\subsubsection{Probability of Success}
		Say that $n=pq$. This algorithm is probabilistic, since we choose $a$
		randomly. So the success of the algorithm depends on the order of $a$
		being even and not having $a^{r/2}\equiv -1\pmod{n}$. Choosing $a$
		randomly in $\mathbb{Z}_n$ is the same as choosing $a_p\in\mathbb{Z}_p$
		and $a_q\in\ints_q$ randomly. Of all elements in $\ints_n^\ast$, half
		have an even order, and half have an odd order. 
		
		Let $r_p$ be the order of $a_p$ and $r_q$ the order of $a_q$. When $r_p$
		and $r_q$
		are even, we know that $2^{k_p}$ divides $r_p$ and $2^{k_q} | r_q$ for
		some integers $k_p$ and $k_q$.
		If $r$ is odd, then both $r_p$ and $r_q$ are odd. This occurs with
		probability $1/2$. When $r$ is even and
		$a^{r/2}\equiv -1\pmod{n}$,
		it follows that $a_p^{r/2}\equiv -1\pmod{p}$ and
		$a_q^{r/2}\equiv-1\pmod{q}$ by the Chinese Remainder Theorem. So $r_p$
		and $r_q$ do not divide $r/2$. Then both $r_p$ and $r_q$ have the same
		largest power of 2 as a divisor, say $2^m | r_p, r_q$. If they do not,
		then say $2^{m+1}|r_p$, and $r$ as well. So $r_q$ would divide $r/2$,
		which is a contradiction.
		So $a$ has an even order with $a^{r/2}\equiv
		-1\pmod{n}$ probability 1/2, since $a$ is randomly chosen.
		
		\subsubsection{Runtime}
		We assume that the quantum part works and runs in polynomial time in the
		size of $n$. Computing the $\gcd$ on lines 2 and 8 takes
		$O((\log{n})^2)$ multiplications. Everything else takes constant time,
		so the runtime of the algorithm is in $O((\log{n})^2)$ ignoring the
		quantum part.
	\subsection{Quantum Part: Period finding}
		Let $\op{U}_f$ be an oracle for a function $f(x) = a^x\pmod{n}$ for a
		fixed $a$. In this case, $a$ is the random number chosen as in Algorithm
		1.
		The oracle is defined such that  $\op{U}_f \ket{i}\ket0 = \ket{i}\ket{f(i)}$.

		We work in the space $\complex^n\tensor\complex^n$
		The following circuit gives the quantum part of Shor's algorithm:
		\[
		\Qcircuit @C=1em @R=.7em {
			\lstick{\ket0}& \qw & \gate{F_n^{\phantom{\dagger}}} & \qw & \multigate{1}{U_f} & \qw &
				\gate{F_n^{\dagger}} &
				\qw & \meter & \qw\\
			\lstick{\ket0}& \qw & \qw & \qw & \ghost{U_f} & \qw & \qw &
		\qw & \meter & \rstick{\ket{z}} \qw
		}
		\]
	\subsubsection{Analysis of the Quantum Part}
		We start by applying the Fourier transform on the first quantum register.
		\[ (\op{F}_n\tensor\op1) \ket0\ket0 =
		\frac{1}{\sqrt{n}}\sum_{i=0}^{n-1} \ket{i}\ket0\]

		Then the first register is in a superposition of all possible states.
		Next we apply the oracle:
		
		\[
			\oracle \frac{1}{\sqrt{n}}\sum_{i=0}^{n-1}\ket{i}\ket0 = 
			\frac{1}{\sqrt{n}}\sum_{i=0}^{n-1}\ket{i}\ket{a^i\mod{n}}
		\]

		When we measure the second register, we get some integer $z$. Let $i_0$
		be the smallest integer with $0\leq i_0 < n$ such that $a^{i_0}\equiv
		z\pmod{n}$. The first register can be rewritten to 
		\[\frac{1}{\sqrt{\floor{n/p}}}\sum_{i=0}^{\lfloor (n/p)-1\rfloor}\ket{i_0+pi},\] 
		where $p$ is the period (order) of $a$. Applying the inverse Fourier
		transform on this gives
		\begin{align*} 
		&\xrightarrow{\op{F}_n^\dagger}\frac1{\sqrt{n}}
		\sum_{j=0}^{n-1}
		\frac{1}{\sqrt{\floor{n/p}}}\sum_{i=0}^{\floor{(n/p)-1}}
		\o{n}{-(j(i_0+pi))}\ket{j}\\
			&= \sum_j\sum_i \o{n}{-ji_0}\o{n}{jpi}\ket{j}\\
			&= \sum_j\o{n}{-ji_0}\sum_i\o{n/p}{ij}\ket{j}
		\end{align*}
		But $\sum_i\o{n/p}{ij}$ is 1 whenever $j$ is a multiple of $n/p$, and 0
		otherwise. The summation becomes
		\[ \sum_{j\mid j \text{ is a multiple of } n/p} \ket{j}\]
		When we measure, we get $j$ such that $j = (n\lambda)/p$ for some
		$\lambda$. We also know $n$, so dividing by it gives $j/n = \lambda/p$.
		Then we can write this fraction in lowest terms and take the denominator
		as $p$. 

		\textbf{Note:} This is a simplified analysis of the algorithm with the
		assumption that the first run gives an irreducible fraction. In reality
		this may not work all of the time, and you would use continued fraction
		expansions to approximate a good value for $p$.

		The quantum part runs in $\widetilde{O}((\log{n})^2)$ gate operations.
		So the total cost of the entire algorithm is around $O((\log{n})^2)$, which is polynomial in the input size.
		\subsection{Factoring Records using Quantum Methods}
		In their paper, Dattani and Bryans\footnote{N. Dattani, N. Bryans, ``Quantum factorization of 56123 with only 4 qubits''} include a table of records for
		quantum factoring. We reproduce it below.
		\begin{table}[H]\centering
			\caption{Quantum Factorization Records}
			\begin{tabular}{c c c c c}\toprule
				Number & \# of factors & \# of qubits needed & Algorithm & Year
				implemented\\\midrule
				15 & 2 & 8 & Shor & 2001\\
				   & 2 & 8 & Shor & 2007\\
				   & 2 & 8 & Shor & 2007\\
				   & 2 & 8 & Shor & 2009\\
				   & 2 & 8 & Shor & 2012\\
				21 & 2 & 10 & Shor & 2012\\
				143 & 2 & 4 & minimization & 2012\\
				56153 & 2 & 4 & minimization & 2012\\\bottomrule
			\end{tabular}
		\end{table}
		Two of the algorithms used are not Shor's, but are included for
		interests' sake.
	\section{Shor's Algorithm for DLP}
		The algorithm is similar to the quantum part for factoring. Instead, 3
		quantum registers are used. We replace the function computed by the
		oracle $\op{U}_f$ to a function $f(x,y) = g^xa^{-y}\pmod{n}$ so that
		$\oracle\ket{x}\ket{y}\ket0 = \ket{x}\ket{y}\ket{f(x,y)}$.
		\[
		\Qcircuit @C=1em @R=.7em {
			\lstick{\ket0}& \qw & \gate{F_n} & \qw & \multigate{2}{U_f} & \qw &
				\gate{F_n^{\dagger}} &
				\qw & \meter & \qw\\
			\lstick{\ket0}& \qw & \gate{F_n} & \qw & \ghost{U_f} & \qw &
				\gate{F_n^{\dagger}} &
				\qw & \meter & \qw\\
			\lstick{\ket0}& \qw & \qw & \qw & \ghost{U_f} & \qw & \qw &
		\qw & \meter & \rstick{\ket{z}} \qw
		}
		\]

	\subsection{Quantum DLP Analysis}

Given a generator $g$, and $g^r\equiv x\pmod{p}$, we want to solve for $r$.
Suppose we know the order of $g$, that is $g^k\equiv 1\pmod{p}$.
Then $g^ax^{-b}\equiv 1 \pmod{p}$ for all $b\ge 0$ where $a=k+br$.

Then $g^{k+br}x^{-b}=g^kg^{br}x^{-b}=g^{r^b}x^{-b}=x^bx^{-b}=x^{b-b}=1$.

%For the analysis, we assume that $q$ can be represented by a polynomial number
%of bits, and $q$ is smooth. Then QFT can be preformed in polynomial time.\\

Beginning with $\ket{0}\ket{0}$ in $\mathbb{C}^p\otimes\mathbb{C}^p$, perform QFT mod $(p-1)$ on the first two qubits.
This gives us the uniform superposition over $a, b$:
\[\frac{1}{p-1}\sum_{a=0,b=0}^{p-2}\ket{a}\ket{b}\]

We then query our oracle to get to the state
\[\frac{1}{p-1}\sum_{a=0,b=0}^{p-2}\ket{a}\ket{b}\ket{g^ax^-b\pmod{p}}.\]

Then we can measure the third qubit. From above, we have $a=k+br$, so we can rewrite the first qubit.

This reduces us to the state
\[\frac{1}{\sqrt{p-1}}\sum_{b=0}^{p-2}\ket{k+br\pmod{p-1}}\ket{b}.\]

We note that $\ket{br+k\pmod{p-1}}\ket{b}$ is a coset of the subgroup generated by $\ket{br \pmod{p-1}}\ket{b}$, so we consider operations on the subgroup instead.

We then have the state
\[\frac{1}{\sqrt{p-1}}\sum_{b=0}^{p-2}\ket{br\pmod{p-1}}\ket{b}.\]

We now perform an inverse QFT mod $(p-1)$ on the first two qubits.

We now take our assumption that $p-1$ is smooth. Then let $\omega$ be the $p-1$ root of unity, that is $\omega=e^{2\pi i/(p-1)}$. Then we get the state:
\[\frac{1}{(p-1)^{3/2}}\sum_{c=0,d=0}^{p-2}\sum_{b=0}^{p-2}(\omega^{rc+d})^b\ket{c}\ket{d}\]

Then if $rc+d \not\equiv 0 \pmod{p-1}$, we are just summing $\omega_{p-1}^1$ to
$\omega_{p-1}^p-1$, which is just $0$. Then we will only get an answer if $rc+d
\equiv 0 \pmod{p-1}$, i.e., $r \equiv -\frac{d}{c} \pmod{p-1}$.

So if $gcd(c,p-1)=1$ we can find $r$. Since otherwise we would be dividing by
zero above.
	
\end{document}
